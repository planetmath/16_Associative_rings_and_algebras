\documentclass[12pt]{article}
\usepackage{pmmeta}
\pmcanonicalname{SigmaDerivation}
\pmcreated{2013-03-22 11:42:50}
\pmmodified{2013-03-22 11:42:50}
\pmowner{antizeus}{11}
\pmmodifier{antizeus}{11}
\pmtitle{sigma derivation}
\pmrecord{22}{30072}
\pmprivacy{1}
\pmauthor{antizeus}{11}
\pmtype{Definition}
\pmcomment{trigger rebuild}
\pmclassification{msc}{16S36}
\pmclassification{msc}{81S40}
\pmclassification{msc}{81Txx}
\pmclassification{msc}{18E05}
\pmclassification{msc}{55N40}
\pmclassification{msc}{18-00}

\endmetadata

\usepackage{amssymb}
\usepackage{amsmath}
\usepackage{amsfonts}
\usepackage{graphicx}
%%%%%%%%%%%%%%\usepackage{xypic}
\begin{document}
If $\sigma$ is a ring endomorphism on a ring $R$, 
then a {\it (left) $\sigma$-derivation}
is an additive map $\delta$ on $R$ such that 
$\delta(x \cdot y)=\sigma(x) \cdot \delta(y) + \delta(x) \cdot y$
for all $x,y$ in $R$.
%%%%%
%%%%%
%%%%%
%%%%%
%%%%%
%%%%%
%%%%%
%%%%%
%%%%%
%%%%%
%%%%%
%%%%%
%%%%%
%%%%%
\end{document}
