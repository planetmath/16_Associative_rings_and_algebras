\documentclass[12pt]{article}
\usepackage{pmmeta}
\pmcanonicalname{FaaDiBrunosFormula}
\pmcreated{2013-03-22 16:38:57}
\pmmodified{2013-03-22 16:38:57}
\pmowner{rspuzio}{6075}
\pmmodifier{rspuzio}{6075}
\pmtitle{Fa\`a di Bruno's formula}
\pmrecord{5}{38853}
\pmprivacy{1}
\pmauthor{rspuzio}{6075}
\pmtype{Definition}
\pmcomment{trigger rebuild}
\pmclassification{msc}{16W30}
\pmsynonym{Faa di Bruno's formula}{FaaDiBrunosFormula}
\pmsynonym{Fa\`a di Bruno formula}{FaaDiBrunosFormula}
\pmsynonym{Faa di Bruno formula}{FaaDiBrunosFormula}

% this is the default PlanetMath preamble.  as your knowledge
% of TeX increases, you will probably want to edit this, but
% it should be fine as is for beginners.

% almost certainly you want these
\usepackage{amssymb}
\usepackage{amsmath}
\usepackage{amsfonts}

% used for TeXing text within eps files
%\usepackage{psfrag}
% need this for including graphics (\includegraphics)
%\usepackage{graphicx}
% for neatly defining theorems and propositions
%\usepackage{amsthm}
% making logically defined graphics
%%%\usepackage{xypic}

% there are many more packages, add them here as you need them

% define commands here

\begin{document}
{\em Fa\`a di Bruno's formula} is a generalization of the chain rule
to higher order derivatives which expresses the derivative of a 
composition of functions as a series of products of derivatives:

$${d^n \over dx^n} f(g(x))=\sum_{\sum_{k=0}^n k m_k = n} \frac{n!}{m_1!\,m_2!\,m_3!\,\cdots 1!^{m_1}\,2!^{m_2}\,3!^{m_3}\,\cdots} f^{(m_1 + \cdots + m_n)}(g(x)) \prod_{j\,:\,m_j\neq 0}\left(g^{(j)}(x)\right)^{m_j}$$

This formula was discovered by Francesco Fa\`a di Bruno in the 1850s and can
be proved by induction on the order of the derivative.

\begin{thebibliography}{1}
\bibitem{} Fa\`a di Bruno, C. F.. ``Sullo sviluppo delle funzione.'' {\it Ann. di 
Scienze Matem. et Fisiche di Tortoloni} {\bf 6} (1855): 479-480
\bibitem{} Fa\`a di Bruno, C. F.. ``Note sur un nouvelle formule de calcul diff\'erentiel.'' {\it Quart. J. Math.} {\bf 1} (1857): 359-360
\bibitem{} H. Figueroa \& J. M. Gracia-Bond\'ia, ``Combinatorial Hopf Algebras in Quantum Field Theory I'' {\it Rev. Math. Phys.} {\bf 17} (2005): 881 - 975
\end{thebibliography}

%%%%%
%%%%%
\end{document}
