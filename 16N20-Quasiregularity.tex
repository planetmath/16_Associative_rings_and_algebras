\documentclass[12pt]{article}
\usepackage{pmmeta}
\pmcanonicalname{Quasiregularity}
\pmcreated{2013-03-22 13:12:59}
\pmmodified{2013-03-22 13:12:59}
\pmowner{mclase}{549}
\pmmodifier{mclase}{549}
\pmtitle{quasi-regularity}
\pmrecord{8}{33681}
\pmprivacy{1}
\pmauthor{mclase}{549}
\pmtype{Definition}
\pmcomment{trigger rebuild}
\pmclassification{msc}{16N20}
\pmsynonym{quasi regular}{Quasiregularity}
\pmsynonym{quasi regularity}{Quasiregularity}
\pmrelated{JacobsonRadical}
\pmrelated{RegularIdeal}
\pmrelated{HomotopesAndIsotopesOfAlgebras}
\pmdefines{quasi-regular}
\pmdefines{right quasi-regular}
\pmdefines{left quasi-regular}
\pmdefines{quasi-inverse}
\pmdefines{quasi-regular ideal}
\pmdefines{quasi-regular ring}

% this is the default PlanetMath preamble.  as your knowledge
% of TeX increases, you will probably want to edit this, but
% it should be fine as is for beginners.

% almost certainly you want these
\usepackage{amssymb}
\usepackage{amsmath}
\usepackage{amsfonts}

% used for TeXing text within eps files
%\usepackage{psfrag}
% need this for including graphics (\includegraphics)
%\usepackage{graphicx}
% for neatly defining theorems and propositions
\usepackage{amsthm}
% making logically defined graphics
%%%\usepackage{xypic}

% there are many more packages, add them here as you need them

% define commands here
\newtheorem*{lemma}{Lemma}
\begin{document}
\PMlinkescapeword{terms}

An element $x$ of a ring is called \emph{right quasi-regular} [resp. \emph{left quasi-regular}] if there
is an element $y$ in the ring such that $x + y + xy = 0$ [resp. $x + y + yx = 0$].

For calculations with quasi-regularity, it is useful to introduce the operation $*$ defined:
$$ x * y = x + y + xy .$$
Thus $x$ is right quasi-regular if there is an element $y$ such that $x * y = 0$.
The operation $*$ is easily demonstrated to be associative, and $x * 0 = 0 * x = x$ for all $x$.

An element $x$ is called \emph{quasi-regular} if it is both left and right quasi-regular.
In this case, there are elements $y$ and $z$ such that $x + y + xy = 0 = x + z + zx$
(equivalently, $x * y = z * x = 0$).
A calculation shows that
$$y = 0 * y = (z * x) * y = z * (x * y) = z.$$
So $y = z$ is a unique element, depending on $x$, called the \emph{quasi-inverse} of $x$.

An ideal (one- or two-sided) of a ring is called \emph{quasi-regular} if each of its elements is quasi-regular.  Similarly, a ring is called \emph{quasi-regular} if each of its elements is quasi-regular (such rings cannot have an identity element).

\begin{lemma}
Let $A$ be an ideal (one- or two-sided) in a ring $R$.
If each element of $A$ is right quasi-regular, then $A$ is a quasi-regular ideal.
\end{lemma}

This lemma means that there is no extra generality gained in defining terms such as right quasi-regular left ideal, etc.

Quasi-regularity is important because it provides elementary characterizations of the Jacobson radical
for rings without an identity element:
\begin{itemize}
\item The Jacobson radical of a ring is the sum of all quasi-regular left (or right) ideals.
\item The Jacobson radical of a ring is the largest quasi-regular ideal of the ring.
\end{itemize}

For rings with an identity element, note that $x$ is [right, left] quasi-regular if and only if $1 + x$ is [right, left]
invertible in the ring.
%%%%%
%%%%%
\end{document}
