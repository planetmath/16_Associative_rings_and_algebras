\documentclass[12pt]{article}
\usepackage{pmmeta}
\pmcanonicalname{ProofOfCharacterizationsOfTheJacobsonRadical}
\pmcreated{2013-03-22 12:48:56}
\pmmodified{2013-03-22 12:48:56}
\pmowner{rspuzio}{6075}
\pmmodifier{rspuzio}{6075}
\pmtitle{proof of characterizations of the Jacobson radical}
\pmrecord{31}{33137}
\pmprivacy{1}
\pmauthor{rspuzio}{6075}
\pmtype{Proof}
\pmcomment{trigger rebuild}
\pmclassification{msc}{16N20}

\endmetadata

% this is the default PlanetMath preamble.  as your knowledge
% of TeX increases, you will probably want to edit this, but
% it should be fine as is for beginners.

% almost certainly you want these
\usepackage{amssymb}
\usepackage{amsmath}
\usepackage{amsfonts}

% used for TeXing text within eps files
%\usepackage{psfrag}
% need this for including graphics (\includegraphics)
%\usepackage{graphicx}
% for neatly defining theorems and propositions
%\usepackage{amsthm}
% making logically defined graphics
%%%\usepackage{xypic}

% there are many more packages, add them here as you need them

% define commands here
\begin{document}
First, note that by definition a left primitive ideal is the annihilator of an irreducible left $R$-module, so clearly characterization 1) is equivalent to the definition of the Jacobson radical.

Next, we will prove cyclical containment.  Observe that 5) follows after the equivalence of 1) - 4) is established, since 4) is independent of the choice of left or right ideals.

\begin{enumerate}
\item [1) $\subset$ 2)] We know that every left primitive ideal is the
      largest ideal contained in a maximal left ideal.  So the intersection
      of all left primitive ideals will be contained in the intersection of all
      maximal left ideals.
\item [2) $\subset$ 3)] Let
      $S=\{M : M \text{ a maximal left ideal of } R\}$ and take $r \in R$.
      Let $t \in \cap_{M \in S} M$.  Then $rt \in \cap_{M \in S} M$.

      Assume $1-rt$ is not left invertible; therefore there exists a maximal
      left ideal $M_0$ of $R$ such that $R(1-rt) \subseteq M_0$.

      Note then that $1-rt \in M_0$.  Also, by definition of $t$, we have
      $rt \in M_0$.  Therefore $1 \in M_0$; this contradiction implies $1-rt$ is
      left invertible.
\item [3) $\subset$ 4)] We claim that 3) satisfies the condition of 4).

      Let $K=\{t \in R : 1-rt \text{ is left invertible for all } r \in R \}$.

      We shall first show that $K$ is an ideal.

      Clearly if $t \in K$, then $rt \in K$.  
      If $t_1,t_2 \in K$, then \[ 1-r(t_1+t_2)=(1-rt_1)-rt_2 \]
      Now there exists $u_1$ such that $u_1(1-rt_1)=1$, hence
      \[ u_1((1-rt_1)-rt_2)=1-u_1rt_2 \]
      Similarly, there exists $u_2$ such that $u_2(1-u_1rt_2)=1$, therefore
      \[ u_2u_1(1-r(t_1+t_2))=1 \]
      Hence $t_1+t_2 \in K$.

      Now if $t \in K, r \in R$, to show that $tr \in K$ it suffices to
      show that $1-tr$ is left invertible.  Suppose $u(1-rt)=1$, hence
      $u-urt=1$, then $tur-turtr=tr$.

      So $(1+tur)(1-tr)=1+tur-tr-turtr=1$.

      Therefore $K$ is an ideal.

      Now let $v \in K$.  Then there exists $u$ such that $u(1-v)=1$, hence
      $1-u=-uv \in K$, so $u=1-(1-u)$ is left invertible.

      So there exists $w$ such that $wu=1$, hence $wu(1-v)=w$, then $1-v=w$.
      Thus $(1-v)u=1$ and therefore $1-v$ is a unit.

      Let $J$ be the largest ideal such that, for all $v \in J$, $1-v$ is a
      unit.  We claim that $K \subseteq J$.

      Suppose this were not true; in this case $K+J$ strictly contains $J$.
      Consider $rx+sy \in K+J$ with $x \in K, y \in J$ and $r,s \in R$.
      Now $1-(rx+sy)=(1-rx)-sy$, and since $rx \in K$, then $1-rx=u$ for some
      unit $u \in R$.

      So $1-(rx+sy)=u-sy=u(1-u^{-1}sy)$, and clearly $u^{-1}sy \in J$ since
      $y \in J$.  Hence $1-u^{-1}sy$ is also a unit, and thus $1-(rx+sy)$ is
      a unit.

      Thus $1-v$ is a unit for all $v \in K+J$.  But this contradicts the
      assumption that $J$ is the largest such ideal.  So we must have
      $K \subseteq J$.
\item [4) $\subset$ 1)] We must show that if $I$ is an ideal such that
      for all $u \in I$, $1-u$ is a unit, then
      $I \subset \operatorname{ann}({}_RM)$ for every irreducible left
      $R$-module ${}_RM$.\\

      Suppose this is not the case, so there exists ${}_RM$ such that
      $I \not \subset \operatorname{ann}({}_RM)$.  Now we know that
      $\operatorname{ann}({}_RM)$ is the largest ideal inside some maximal
      left ideal $J$ of $R$.  Thus we must also have $I \not \subset J$,
      or else this would contradict the maximality of
      $\operatorname{ann}({}_RM)$ inside $J$.

      But since $I \not \subset J$, then by maximality $I+J=R$, hence there
      exist $u \in I$ and $v \in J$ such that $u+v=1$.  Then $v=1-u$, so
      $v$ is a unit and $J=R$.  But since $J$ is a proper left ideal, this
      is a contradiction.
\end{enumerate}
%%%%%
%%%%%
\end{document}
