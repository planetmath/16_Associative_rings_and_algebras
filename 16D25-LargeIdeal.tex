\documentclass[12pt]{article}
\usepackage{pmmeta}
\pmcanonicalname{LargeIdeal}
\pmcreated{2013-03-22 15:37:28}
\pmmodified{2013-03-22 15:37:28}
\pmowner{jocaps}{12118}
\pmmodifier{jocaps}{12118}
\pmtitle{large ideal}
\pmrecord{12}{37549}
\pmprivacy{1}
\pmauthor{jocaps}{12118}
\pmtype{Definition}
\pmcomment{trigger rebuild}
\pmclassification{msc}{16D25}

\endmetadata

% this is the default PlanetMath preamble.  as your knowledge
% of TeX increases, you will probably want to edit this, but
% it should be fine as is for beginners.

% almost certainly you want these
\usepackage{amssymb}
\usepackage{amsmath}
\usepackage{amsfonts}

% used for TeXing text within eps files
%\usepackage{psfrag}
% need this for including graphics (\includegraphics)
%\usepackage{graphicx}
% for neatly defining theorems and propositions
%\usepackage{amsthm}
% making logically defined graphics
%%%\usepackage{xypic}

% there are many more packages, add them here as you need them

% define commands here
\begin{document}
An ideal $I$ of a ring $R$ is called a \emph{large ideal} if for every ideal $J$ of $R$ such that $J\neq\{0\}$, $I\cap J \neq\{0\}$

A ring is semiprime iff every large ideal is dense.

Obviously all nontrivial ideal of an integral domain is a large ideal, and the maximal ideal of any non-trivial local ring is a large ideal.

\begin{thebibliography}{99}
\bibitem{roqrof}
\textbf{N.J. Fine, L. Gillman, J. Lambek}, 
"Rings of Quotients of Rings of Functions",\\
Transcribed and edited into PDF from the original 1966 McGill University Press book \\
(see \PMlinkexternal{here}{http://tinyurl.com/24unqs}, Editors: M. Barr, R. Raphael),\\
\PMlinkexternal{Online download}{http://tinyurl.com/ytw3tj},
Accessed 24.10.2007
\end{thebibliography}
%%%%%
%%%%%
\end{document}
