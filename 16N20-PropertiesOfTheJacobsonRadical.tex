\documentclass[12pt]{article}
\usepackage{pmmeta}
\pmcanonicalname{PropertiesOfTheJacobsonRadical}
\pmcreated{2013-03-22 12:49:43}
\pmmodified{2013-03-22 12:49:43}
\pmowner{yark}{2760}
\pmmodifier{yark}{2760}
\pmtitle{properties of the Jacobson radical}
\pmrecord{12}{33152}
\pmprivacy{1}
\pmauthor{yark}{2760}
\pmtype{Result}
\pmcomment{trigger rebuild}
\pmclassification{msc}{16N20}

% this is the default PlanetMath preamble.  as your knowledge
% of TeX increases, you will probably want to edit this, but
% it should be fine as is for beginners.

% almost certainly you want these
\usepackage{amssymb}
\usepackage{amsmath}
\usepackage{amsfonts}

% used for TeXing text within eps files
%\usepackage{psfrag}
% need this for including graphics (\includegraphics)
%\usepackage{graphicx}
% for neatly defining theorems and propositions
%\usepackage{amsthm}
% making logically defined graphics
%%%\usepackage{xypic}

% there are many more packages, add them here as you need them

% define commands here
\begin{document}
{\bf Theorem:}\\
Let $R, T$ be rings and $\varphi : R \rightarrow T$ be a surjective homomorphism.  Then $\varphi(J(R)) \subseteq J(T)$.

{\bf Proof:}\\
We shall use the characterization of the Jacobson radical as the set of all $a \in R$ such that for all $r \in R$, $1-ra$ is left invertible.

Let $a \in J(R), t \in T$.  We claim that $1-t\varphi(a)$ is left invertible:

Since $\varphi$ is surjective, $t=\varphi(r)$ for some $r \in R$.  Since $a \in J(R)$, we know $1-ra$ is left invertible, so there exists $u \in R$ such that
$u(1-ra)=1$.  Then we have
\[ \varphi(u) \left(
   \varphi(1)-\varphi(r)\varphi(a) \right)
    = \varphi(u)\varphi(1-ra) = \varphi(1)=1 \]
So $\varphi(a) \in J(T)$ as required.\\

{\bf Theorem:}\\
Let $R, T$ be rings.  Then $J(R \times T) \subseteq J(R) \times J(T)$.

{\bf Proof:}\\
Let $\pi_1 : R \times T \rightarrow R$ be a (surjective) projection.
By the previous theorem, $\pi_1(J(R \times T)) \subseteq J(R)$.\\

Similarly let $\pi_2 : R \times T \rightarrow T$ be a (surjective) projection.  We see that $\pi_2(J(R \times T)) \subseteq J(T)$.\\

Now take $(a,b) \in J(R \times T)$.  Note that $a = \pi_1(a,b) \in J(R)$ and $b = \pi_2(a,b) \in J(T)$.  Hence $(a,b) \in J(R) \times J(T)$ as required.
%%%%%
%%%%%
\end{document}
