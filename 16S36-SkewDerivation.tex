\documentclass[12pt]{article}
\usepackage{pmmeta}
\pmcanonicalname{SkewDerivation}
\pmcreated{2013-03-22 11:49:21}
\pmmodified{2013-03-22 11:49:21}
\pmowner{antizeus}{11}
\pmmodifier{antizeus}{11}
\pmtitle{skew derivation}
\pmrecord{9}{30369}
\pmprivacy{1}
\pmauthor{antizeus}{11}
\pmtype{Definition}
\pmcomment{trigger rebuild}
\pmclassification{msc}{16S36}
\pmrelated{SigmaDerivation}

\endmetadata

\usepackage{amssymb}
\usepackage{amsmath}
\usepackage{amsfonts}
\usepackage{graphicx}
%%%%\usepackage{xypic}
\begin{document}
A {\it (left) skew derivation}
on a ring $R$ is a pair $(\sigma, \delta)$,
where $\sigma$ is a ring endomorphism of $R$,
and $\delta$ is a left \PMlinkname{$\sigma$-derivation}{SigmaDerivation} on $R$.
%%%%%
%%%%%
%%%%%
%%%%%
\end{document}
