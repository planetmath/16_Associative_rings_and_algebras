\documentclass[12pt]{article}
\usepackage{pmmeta}
\pmcanonicalname{CentralSimpleAlgebra}
\pmcreated{2013-03-22 11:49:08}
\pmmodified{2013-03-22 11:49:08}
\pmowner{djao}{24}
\pmmodifier{djao}{24}
\pmtitle{central simple algebra}
\pmrecord{7}{30363}
\pmprivacy{1}
\pmauthor{djao}{24}
\pmtype{Definition}
\pmcomment{trigger rebuild}
\pmclassification{msc}{16D60}
\pmclassification{msc}{70K75}

\usepackage{amssymb}
\usepackage{amsmath}
\usepackage{amsfonts}
\usepackage{graphicx}
%%%%\usepackage{xypic}
\begin{document}
Let $K$ be a field. A {\em central simple algebra} $A$ (over $K$) is an algebra $A$ over $K$, which is finite dimensional as a vector space over $K$, such that
\begin{itemize}
\item $A$ has an identity element, as a ring
\item $A$ is central: the center of $A$ equals $K$ (for all $z \in A$, we have $z\cdot a = a \cdot z$ for all $a \in A$ if and only if $z \in K$)
\item $A$ is simple: for any two sided ideal $I$ of $A$, either $I = \{0\}$ or $I = A$
\end{itemize}

By a theorem of Brauer, for every central simple algebra $A$ over $K$, there exists a unique (up to isomorphism) division ring $D$ containing $K$ and a unique natural number $n$ such that $A$ is isomorphic to the ring of $n \times n$ matrices with coefficients in $D$.
%%%%%
%%%%%
%%%%%
%%%%%
\end{document}
