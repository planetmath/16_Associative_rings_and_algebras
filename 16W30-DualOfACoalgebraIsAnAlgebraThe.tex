\documentclass[12pt]{article}
\usepackage{pmmeta}
\pmcanonicalname{DualOfACoalgebraIsAnAlgebraThe}
\pmcreated{2013-03-22 16:34:20}
\pmmodified{2013-03-22 16:34:20}
\pmowner{mps}{409}
\pmmodifier{mps}{409}
\pmtitle{dual of a coalgebra is an algebra, the}
\pmrecord{8}{38762}
\pmprivacy{1}
\pmauthor{mps}{409}
\pmtype{Derivation}
\pmcomment{trigger rebuild}
\pmclassification{msc}{16W30}
%\pmkeywords{Dualities of al/gebras and coalgebras}
%\pmkeywords{mirror or tangled duality vs. categorical}
%\pmkeywords{arrow reversing duality and factorization}
\pmrelated{GrassmanHopfAlgebrasAndTheirDualCoAlgebras}
\pmrelated{DualityInMathematics}
\pmrelated{QuantumGroups}
\pmdefines{dualities of algebraic structures}

\endmetadata

% this is the default PlanetMath preamble.  as your knowledge
% of TeX increases, you will probably want to edit this, but
% it should be fine as is for beginners.

% almost certainly you want these
\usepackage{amssymb}
\usepackage{amsmath}
\usepackage{amsfonts}

% used for TeXing text within eps files
%\usepackage{psfrag}
% need this for including graphics (\includegraphics)
%\usepackage{graphicx}
% for neatly defining theorems and propositions
\usepackage{amsthm}
% making logically defined graphics
%%%\usepackage{xypic}

% there are many more packages, add them here as you need them

% define commands here
\newcommand{\Hom}[3][]{\mathrm{Hom}_{#1}({#2},{#3})}

\theoremstyle{definition}
\newtheorem*{example*}{Example}
\begin{document}
\PMlinkescapeword{convolution}
\PMlinkescapeword{product}
\PMlinkescapeword{similar}

Let $R$ be a commutative ring with unity.  Suppose we have a coassociative coalgebra $(C,\Delta)$ and an associative algebra $A$, both over $R$.  Since $C$ and $A$ are both $R$-modules, it follows that $\Hom[R]{C}{A}$ is also an $R$-module.  But in fact we can give it the structure of an associative $R$-algebra.  To do this, we use the convolution product.  Namely, given morphisms $f$ and $g$ in $\Hom[R]{C}{A}$, we define their product $fg$ by
\[
(fg)(x) = \sum_x f(x_{(1)})\cdot g(x_{(2)}),
\]
where we use the Sweedler notation 
\[
\Delta(x) = \sum_x x_{(1)}\otimes x_{(2)}
\]
for the comultiplication $\Delta$.  To see that the convolution product is associative, suppose $f$, $g$, and $h$ are in $\Hom[R]{C}{A}$.  By applying the coassociativity of $\Delta$, we may write
\[
((fg)h)(x) = \sum_x (f(x_{(1)})\cdot g(x_{(2)}))\cdot h(x_{(3)})
\]
and
\[
(f(gh))(x) = \sum_x f(x_{(1)})\cdot (g(x_{(2)}))\cdot h(x_{(3)}).
\]
Since $A$ has an associative product, it follows that $(fg)h=f(gh)$.

In the foregoing, we have not assumed that $C$ is counitary or that $A$ is unitary.  If $C$ is counitary with counit $\varepsilon\colon C\to R$ and $A$ is unitary with identity $1\colon R\to A$, then their composition $1\circ\varepsilon\colon C\to A$ is the identity for the convolution product.

\begin{example*}
Let $C$ be a coassociative coalgebra over $R$.  Then $R$ itself is an associative $R$-algebra.  The algebra $\Hom[R]{C}{R}$ is called the \emph{algebra dual to the coalgebra $C$}.
\end{example*}

We have seen that any coalgebra dualizes to give an algebra.  One might expect that a similar construction could be performed on $\Hom[R]{A}{R}$ to give a coalgebra dual to $A$.  However, this is not the case.  Thus coalgebras (based on ``factoring'') are more fundamental than algebras (based on ``multiplying'').

(The proof will be provided at a later stage).

\textbf{Remark on Al/gebraic Duality}--{\em Mirror or tangled `duality' of algebras and `gebras':} \\
An interesting twist to duality was provided in Fauser's publications on al/gebras
where mirror or tangled `duality' has been defined for Grassman-Hopf al/gebras. Thus,
an algebra not only has the usual reversed arrow dual coalgebra but a mirror (or tangled)
gebra which is quite distinct from the coalgebra.

\textbf{Note:}
The dual of a quantum group is a Hopf algebra.   


\begin{thebibliography}{99}
\bibitem{NiSw}
W.~Nichols and M.~Sweedler, {\it Hopf algebras and combinatorics}, in {\it Proceedings of the conference on umbral calculus and Hopf algebras}, ed. R.~Morris, AMS, 1982.

\bibitem{Fauser2002}
B. Fauser: \emph{A treatise on quantum Clifford Algebras}. Konstanz,
Habilitationsschrift. \\ arXiv.math.QA/0202059 (2002).

\bibitem{Fauser2004}
B. Fauser: Grade Free product Formulae from Grassman--Hopf Gebras.
Ch. 18 in R. Ablamowicz, Ed., \emph{Clifford Algebras: Applications to Mathematics, Physics and Engineering}, Birkh\"{a}user: Boston, Basel and Berlin, (2004).

\bibitem{Fell}
J. M. G. Fell.: The Dual Spaces of  C*--Algebras., \emph{Transactions of the American
Mathematical Society}, \textbf{94}: 365--403 (1960).
\end{thebibliography}
%%%%%
%%%%%
\end{document}
