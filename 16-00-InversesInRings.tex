\documentclass[12pt]{article}
\usepackage{pmmeta}
\pmcanonicalname{InversesInRings}
\pmcreated{2013-03-22 17:08:55}
\pmmodified{2013-03-22 17:08:55}
\pmowner{Wkbj79}{1863}
\pmmodifier{Wkbj79}{1863}
\pmtitle{inverses in rings}
\pmrecord{4}{39457}
\pmprivacy{1}
\pmauthor{Wkbj79}{1863}
\pmtype{Topic}
\pmcomment{trigger rebuild}
\pmclassification{msc}{16-00}
\pmrelated{Klein4Ring}
\pmrelated{LeftAndRightUnityOfRing}
\pmdefines{left invertible}
\pmdefines{right invertible}
\pmdefines{left inverse}
\pmdefines{right inverse}

\endmetadata

\usepackage{amssymb}
\usepackage{amsmath}
\usepackage{amsfonts}
\usepackage{pstricks}
\usepackage{psfrag}
\usepackage{graphicx}
\usepackage{amsthm}
%%\usepackage{xypic}

\begin{document}
Let $R$ be a ring with unity $1$ and $r \in R$.  Then $r$ is \emph{left invertible} if there exists $q \in R$ with $qr=1$; $q$ is a \emph{left inverse} of $r$.  Similarly, $r$ is \emph{right invertible} if there exists $s \in R$ with $rs=1$; $s$ is a \emph{right inverse} of $r$.

Note that, if $r$ is left invertible, it may not have a unique left inverse, and similarly for right invertible elements.  On the other hand, if $r$ is left invertible and right invertible, then it has exactly one left inverse and one right inverse.  Moreover, these two \PMlinkescapetext{inverses} are equal, and $r$ is a unit.
%%%%%
%%%%%
\end{document}
