\documentclass[12pt]{article}
\usepackage{pmmeta}
\pmcanonicalname{ProductOfIdeals}
\pmcreated{2013-03-22 11:50:59}
\pmmodified{2013-03-22 11:50:59}
\pmowner{mathcam}{2727}
\pmmodifier{mathcam}{2727}
\pmtitle{product of ideals}
\pmrecord{11}{30411}
\pmprivacy{1}
\pmauthor{mathcam}{2727}
\pmtype{Definition}
\pmcomment{trigger rebuild}
\pmclassification{msc}{16D25}
\pmclassification{msc}{15A15}
\pmclassification{msc}{46L87}
\pmclassification{msc}{55U40}
\pmclassification{msc}{55U35}
\pmclassification{msc}{81R10}
\pmclassification{msc}{46L05}
\pmclassification{msc}{22A22}
\pmclassification{msc}{81R50}
\pmclassification{msc}{18B40}
\pmrelated{SumOfIdeals}
\pmrelated{QuotientOfIdeals}
\pmrelated{PruferRing}
\pmrelated{ProductOfLeftAndRightIdeal}
\pmrelated{WellDefinednessOfProductOfFinitelyGeneratedIdeals}

\endmetadata

\usepackage{amssymb}
\usepackage{amsmath}
\usepackage{amsfonts}
\usepackage{graphicx}
%%%%\usepackage{xypic}
\begin{document}
Let $R$ be a ring, and let $A$ and $B$ be left (right) ideals of $R$.  Then the {\it product} of the ideals $A$ and $B$, which we denote $AB$, is the left (right) ideal generated by all products $ab$ with $a\in A$ and $b\in B$.  Note that since sums of products of the form $ab$ with $a\in A$ and $b\in B$ are contained simultaneously in both $A$ and $B$, we have $AB\subset A\cap B$.
%%%%%
%%%%%
%%%%%
%%%%%
\end{document}
