\documentclass[12pt]{article}
\usepackage{pmmeta}
\pmcanonicalname{UniformModule}
\pmcreated{2013-03-22 11:51:20}
\pmmodified{2013-03-22 11:51:20}
\pmowner{antizeus}{11}
\pmmodifier{antizeus}{11}
\pmtitle{uniform module}
\pmrecord{8}{30418}
\pmprivacy{1}
\pmauthor{antizeus}{11}
\pmtype{Definition}
\pmcomment{trigger rebuild}
\pmclassification{msc}{16D80}
\pmdefines{uniform submodule}

\endmetadata

\usepackage{amssymb}
\usepackage{amsmath}
\usepackage{amsfonts}
\usepackage{graphicx}
%%%%\usepackage{xypic}
\begin{document}
A module $M$ is said to be {\it uniform} if any two nonzero submodules of $M$ must have a nonzero intersection.  This is equivalent to saying that any nonzero submodule is an essential submodule.
%%%%%
%%%%%
%%%%%
%%%%%
\end{document}
