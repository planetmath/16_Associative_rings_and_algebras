\documentclass[12pt]{article}
\usepackage{pmmeta}
\pmcanonicalname{PID}
\pmcreated{2013-03-22 11:56:25}
\pmmodified{2013-03-22 11:56:25}
\pmowner{mps}{409}
\pmmodifier{mps}{409}
\pmtitle{PID}
\pmrecord{13}{30674}
\pmprivacy{1}
\pmauthor{mps}{409}
\pmtype{Definition}
\pmcomment{trigger rebuild}
\pmclassification{msc}{16D25}
\pmclassification{msc}{13G05}
\pmclassification{msc}{11N80}
\pmclassification{msc}{13A15}
\pmsynonym{principal ideal domain}{PID}
\pmrelated{UFD}
\pmrelated{Irreducible}
\pmrelated{Ideal}
\pmrelated{IntegralDomain}
\pmrelated{EuclideanRing}
\pmrelated{EuclideanValuation}
\pmrelated{ProofThatAnEuclideanDomainIsAPID}
\pmrelated{WhyEuclideanDomains}

\endmetadata

\usepackage{amssymb}
\usepackage{amsmath}
\usepackage{amsfonts}
\usepackage{graphicx}
%%%%\usepackage{xypic}
\begin{document}
A \emph{principal ideal domain} is an integral domain where every
ideal is a principal ideal.

In a PID, an ideal $(p)$ is maximal if and only if $p$ is irreducible
(and prime since \PMlinkname{any PID is also a UFD}{PIDsAreUFDs}).

Note that subrings of PIDs are not necessarily PIDs.  (There is
an example of this within the entry biquadratic field.)

%%%%%
%%%%%
%%%%%
%%%%%
\end{document}
