\documentclass[12pt]{article}
\usepackage{pmmeta}
\pmcanonicalname{SubdirectProductOfRings}
\pmcreated{2013-03-22 14:19:11}
\pmmodified{2013-03-22 14:19:11}
\pmowner{CWoo}{3771}
\pmmodifier{CWoo}{3771}
\pmtitle{subdirect product of rings}
\pmrecord{15}{35786}
\pmprivacy{1}
\pmauthor{CWoo}{3771}
\pmtype{Definition}
\pmcomment{trigger rebuild}
\pmclassification{msc}{16D70}
\pmclassification{msc}{16S60}
\pmsynonym{subdirect sum}{SubdirectProductOfRings}
\pmdefines{trivial subdirect product}

% this is the default PlanetMath preamble.  as your knowledge
% of TeX increases, you will probably want to edit this, but
% it should be fine as is for beginners.

% almost certainly you want these
\usepackage{amssymb}
\usepackage{amsmath}
\usepackage{amsfonts}

% used for TeXing text within eps files
%\usepackage{psfrag}
% need this for including graphics (\includegraphics)
%\usepackage{graphicx}
% for neatly defining theorems and propositions
%\usepackage{amsthm}
% making logically defined graphics
%%%\usepackage{xypic}

% there are many more packages, add them here as you need them

% define commands here
\begin{document}
A ring $R$ is said to be (represented as) a \emph{subdirect product} of a family of rings $\lbrace R_i: i \in I \rbrace$ if:
\begin{enumerate}
\item
there is a monomorphism $\varepsilon: R \longrightarrow \prod R_i$, and
\item
given 1., $\pi_i \circ \varepsilon: R \longrightarrow R_i$ is surjective for each $i \in I$, where $\pi_i:\prod R_i \longrightarrow R_i$ is the canonical projection map.
\end{enumerate}

A \emph{subdirect product} (\PMlinkescapetext{representation}) of $R$ is said to be \emph{trivial} if one of the $\pi_i \circ \varepsilon: R \longrightarrow R_i$ is an isomorphism.

Direct products and direct sums of rings are all examples of subdirect products of rings.  $\mathbb{Z}$ does not have non-trivial direct product nor non-trivial direct sum \PMlinkescapetext{representations} of rings.  However, $\mathbb{Z}$ can be represented as a non-trivial subdirect product of $\mathbb{Z}/({p_i}^{n_i})$.

As an application of subdirect products, it can be shown that any ring can be represented as a subdirect product of subdirectly irreducible rings.  Since a subdirectly \PMlinkescapetext{irreducible} commutative reduced ring is a field, a Boolean ring $B$ can be represented as a subdirect product of $\mathbb{Z}_2$.  Furthermore, if this Boolean ring $B$ is finite, the subdirect product \PMlinkescapetext{representation} becomes a direct product \PMlinkescapetext{representation}.  Consequently, $B$ has $2^n$ elements, where $n$ is the number of copies of $\mathbb{Z}_2$.
%%%%%
%%%%%
\end{document}
