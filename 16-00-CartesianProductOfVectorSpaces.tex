\documentclass[12pt]{article}
\usepackage{pmmeta}
\pmcanonicalname{CartesianProductOfVectorSpaces}
\pmcreated{2013-03-22 15:16:06}
\pmmodified{2013-03-22 15:16:06}
\pmowner{Mathprof}{13753}
\pmmodifier{Mathprof}{13753}
\pmtitle{Cartesian product of vector spaces}
\pmrecord{8}{37055}
\pmprivacy{1}
\pmauthor{Mathprof}{13753}
\pmtype{Definition}
\pmcomment{trigger rebuild}
\pmclassification{msc}{16-00}
\pmclassification{msc}{13-00}
\pmclassification{msc}{20-00}
\pmclassification{msc}{15-00}

\endmetadata

% this is the default PlanetMath preamble.  as your knowledge
% of TeX increases, you will probably want to edit this, but
% it should be fine as is for beginners.

% almost certainly you want these
\usepackage{amssymb}
\usepackage{amsmath}
\usepackage{amsfonts}
\usepackage{amsthm}

\usepackage{mathrsfs}

% used for TeXing text within eps files
%\usepackage{psfrag}
% need this for including graphics (\includegraphics)
%\usepackage{graphicx}
% for neatly defining theorems and propositions
%
% making logically defined graphics
%%%\usepackage{xypic}

% there are many more packages, add them here as you need them

% define commands here

\newcommand{\sR}[0]{\mathbb{R}}
\newcommand{\sC}[0]{\mathbb{C}}
\newcommand{\sN}[0]{\mathbb{N}}
\newcommand{\sZ}[0]{\mathbb{Z}}

 \usepackage{bbm}
 \newcommand{\Z}{\mathbbmss{Z}}
 \newcommand{\C}{\mathbbmss{C}}
 \newcommand{\F}{\mathbbmss{F}}
 \newcommand{\R}{\mathbbmss{R}}
 \newcommand{\Q}{\mathbbmss{Q}}



\newcommand*{\norm}[1]{\lVert #1 \rVert}
\newcommand*{\abs}[1]{| #1 |}



\newtheorem{thm}{Theorem}
\newtheorem{defn}{Definition}
\newtheorem{prop}{Proposition}
\newtheorem{lemma}{Lemma}
\newtheorem{cor}{Corollary}
\begin{document}
Suppose $V_1,\ldots, V_N$ are vector spaces over a field $\F$.
Then the Cartesian product $V_1\times \cdots \times V_N$ is a vector space
when addition and scalar multiplication is  defined as follows
\begin{eqnarray*}
   (u_1,\ldots, u_N) +    (v_1,\ldots, v_N) &=&    (u_1+v_1,\ldots, u_N+v_N), \\
   k (u_1,\ldots, u_N) &=& (k u_1,\ldots, k u_N)
\end{eqnarray*}
for $u_i, v_i \in V_i$, $k\in \F$.

For example, the vector space structure of $\R^n$ if defined as above.

\subsubsection*{Properties}
\begin{enumerate}
\item If $V_i$ are vector spaces and $W_i\subset V_i$ are subspaces,
    then $W_1\times \cdots \times W_N$ is a vector subspace of 
    $V_1\times \cdots \times V_N$.
\item The dimension of $V_1\times \cdots \times V_N$ is 
    $\dim V_1+ \cdots +\dim V_N$.
\end{enumerate}
%%%%%
%%%%%
\end{document}
