\documentclass[12pt]{article}
\usepackage{pmmeta}
\pmcanonicalname{AssociatedPrime}
\pmcreated{2013-03-22 12:01:37}
\pmmodified{2013-03-22 12:01:37}
\pmowner{rspuzio}{6075}
\pmmodifier{rspuzio}{6075}
\pmtitle{associated prime}
\pmrecord{10}{31001}
\pmprivacy{1}
\pmauthor{rspuzio}{6075}
\pmtype{Definition}
\pmcomment{trigger rebuild}
\pmclassification{msc}{16D25}
\pmsynonym{annihilator prime}{AssociatedPrime}

\endmetadata

\usepackage{amssymb}
\usepackage{amsmath}
\usepackage{amsfonts}
\usepackage{graphicx}
%%%\usepackage{xypic}
\begin{document}
Let $R$ be a ring, 
and let $M$ be an $R$-module.
A prime ideal $P$ of $R$ 
is an {\PMlinkescapetext {\it annihilator prime}} for $M$
if $P={\rm ann}(X)$, the annihilator of some nonzero submodule $X$ of $M$.

Note that if this is the case, then the module ${\rm ann}_M(P)$ contains $X$, has $P$ as its annihilator,
and is a \PMlinkname{faithful}{FaithfulModule} $(R/P)$-module.

If, in addition, $P$ is equal to the annihilator of a submodule of $M$ that is a \PMlinkname{fully faithful}{FaithfulModule} $(R/P)$-module, then we call $P$ an {\PMlinkescapetext {\it associated prime}} of $M$.
%%%%%
%%%%%
%%%%%
\end{document}
