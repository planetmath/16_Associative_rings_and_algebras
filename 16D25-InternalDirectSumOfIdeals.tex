\documentclass[12pt]{article}
\usepackage{pmmeta}
\pmcanonicalname{InternalDirectSumOfIdeals}
\pmcreated{2013-03-22 14:49:32}
\pmmodified{2013-03-22 14:49:32}
\pmowner{Mathprof}{13753}
\pmmodifier{Mathprof}{13753}
\pmtitle{internal direct sum of ideals}
\pmrecord{7}{36489}
\pmprivacy{1}
\pmauthor{Mathprof}{13753}
\pmtype{Theorem}
\pmcomment{trigger rebuild}
\pmclassification{msc}{16D25}
\pmclassification{msc}{11N80}
\pmclassification{msc}{13A15}
\pmdefines{internal direct sum of ideals}

% this is the default PlanetMath preamble.  as your knowledge
% of TeX increases, you will probably want to edit this, but
% it should be fine as is for beginners.

% almost certainly you want these
\usepackage{amssymb}
\usepackage{amsmath}
\usepackage{amsfonts}

% used for TeXing text within eps files
%\usepackage{psfrag}
% need this for including graphics (\includegraphics)
%\usepackage{graphicx}
% for neatly defining theorems and propositions
 \usepackage{amsthm}
% making logically defined graphics
%%%\usepackage{xypic}

% there are many more packages, add them here as you need them

% define commands here

\theoremstyle{definition}
\newtheorem*{thmplain}{Theorem}
\begin{document}
Let $R$ be a ring and $\mathfrak{a}_1$, $\mathfrak{a}_2$, \ldots, $\mathfrak{a}_n$ its ideals (left, right or two-sided). \,We say that $R$ is the {\em internal direct sum} of these ideals, denoted by
  $$R =  \mathfrak{a}_1\oplus\mathfrak{a}_2\oplus\cdots\oplus\mathfrak{a}_n,$$
if both of the following conditions are true:
$$R = \mathfrak{a}_1+\mathfrak{a}_2+\cdots+\mathfrak{a}_n,$$
$$\mathfrak{a}_i\cap\sum_{j\neq i}\mathfrak{a}_j = \{0\}\quad\forall i.$$

\begin{thmplain}
 \, If  $\mathfrak{a}_1$, $\mathfrak{a}_2$, \ldots, $\mathfrak{a}_n$ are ideals of the ring $R$, then the following two statements are equivalent:
\begin{itemize}
\item $R = \mathfrak{a}_1\oplus\mathfrak{a}_2\oplus\cdots\oplus\mathfrak{a}_n$.
\item Every element $r$ of $R$ has a unique expression\\    
$r = a_1\!+\!a_2\!+\cdots+\!a_n$ \,\,with \,$a_i\in\mathfrak{a}_i\,\,\,\forall i$.
\end{itemize}
\end{thmplain}
%%%%%
%%%%%
\end{document}
