\documentclass[12pt]{article}
\usepackage{pmmeta}
\pmcanonicalname{ExamplesOfRings}
\pmcreated{2013-03-22 15:00:42}
\pmmodified{2013-03-22 15:00:42}
\pmowner{matte}{1858}
\pmmodifier{matte}{1858}
\pmtitle{examples of rings}
\pmrecord{42}{36718}
\pmprivacy{1}
\pmauthor{matte}{1858}
\pmtype{Example}
\pmcomment{trigger rebuild}
\pmclassification{msc}{16-00}
\pmclassification{msc}{13-00}
\pmrelated{CommutativeRing}
\pmrelated{Ring}

\usepackage{amssymb}
\usepackage{amsmath}
\usepackage{amsfonts}
\usepackage{amsthm}
\usepackage{mathrsfs}

\newcommand{\sR}[0]{\mathbb{R}}
\newcommand{\sC}[0]{\mathbb{C}}
\newcommand{\sN}[0]{\mathbb{N}}
\newcommand{\sZ}[0]{\mathbb{Z}}
\newcommand{\sQ}[0]{\mathbb{Q}}
\newcommand{\sH}[0]{\mathbb{H}}

 \usepackage{bbm}
 \newcommand{\Z}{\mathbbmss{Z}}
 \newcommand{\C}{\mathbbmss{C}}
 \newcommand{\R}{\mathbbmss{R}}
 \newcommand{\Q}{\mathbbmss{Q}}

\DeclareMathOperator{\End}{End}

\newcommand*{\norm}[1]{\lVert #1 \rVert}
\newcommand*{\abs}[1]{| #1 |}

\newtheorem{thm}{Theorem}
\newtheorem{defn}{Definition}
\newtheorem{prop}{Proposition}
\newtheorem{lemma}{Lemma}
\newtheorem{cor}{Corollary}
\begin{document}
Rings in this article are assumed to have a commutative addition
with negatives and an associative multiplication.  However, it
is not generally assumed that all rings included here are unital.

\subsubsection*{Examples of commutative rings}

\begin{enumerate}
\item the zero ring,
\item the ring of integers $\sZ$,
\item the ring of even integers $2\sZ$ (a ring without identity), or more generally, $n\sZ$ for any integer $n$,
\item the \PMlinkname{integers modulo $n$}{MathbbZ_n}, $\sZ/n\sZ$,
\item the ring of integers $\mathcal{O}_K$ of a number field $K$,
\item the \PMlinkname{$p$-integral rational numbers}{PAdicValuation} (where $p$ is a prime number),
\item other rings of rational numbers
\item the \PMlinkname{$p$-adic integers}{PAdicIntegers} $\sZ_p$ and the $p$-adic numbers $\sQ_p$,
\item the rational numbers $\sQ$, 
\item the real numbers $\sR$, 
\item rings and fields of algebraic numbers,
\item the complex numbers $\sC$,
\item The set $2^{A}$ of all subsets of a set $A$ is a ring.  The addition is the symmetric difference ``$\triangle$'' and the multiplication the set operation intersection ``$\cap$''.  Its additive identity is the empty set $\varnothing$, and its multiplicative identity is the set $A$.  This is an example of a Boolean ring.
\end{enumerate}

\subsubsection*{Examples of non-commutative rings}
\begin{enumerate}
\item the quaternions, $\sH$, also known as the Hamiltonions.  This is a finite dimensional division ring 
over the real numbers, but noncommutative.  
\item the set of square matrices $M_n(R)$, with $n>1$,
\item the set of triangular matrices (upper or lower, but not both in the same set), 
\item \PMlinkname{strict triangular matrices}{StrictUpperTriangularMatrix} (same condition as above),
\item Klein 4-ring,
\item Let $A$ be an abelian group.  Then the set of group endomorphisms $f:A\to A$ forms a ring $\End A$, 
with addition defined elementwise ($(f+g)(a)=f(a)+g(a)$) and multiplication the functional composition.  
It is the ring of operators over $A$.

By contrast, the set of all functions $\{f:A\to A\}$ are closed to addition and composition, however,
there are generally functions $f$ such that $f\circ(g+h)\neq f\circ g+f\circ g$ and so this set
forms only a near ring.
\end{enumerate}



\subsubsection*{Change of rings (rings generated from other rings)}
Let $R$ be a ring.
\begin{enumerate}
\item If $I$ is an ideal of $R$, then the quotient $R/I$ is a ring, called a quotient ring.
\item $R[x]$ is the polynomial ring over $R$ in one indeterminate $x$ (or alternatively, one can think that $R[x]$ is any transcendental extension ring of $R$, such as $\mathbb{Z}[\pi]$ is over $\mathbb{Z}$),
\item $R(x)$ is the field of rational functions in $x$,
\item $R[[x]]$ is the ring of formal power series in $x$,
\item $R((x))$ is the ring of formal Laurent series in $x$,
\item $M_{n\times n}(R)$ is the $n\times n$ matrix ring over $R$.
\item A special case of Example 6 under the section on non-commutative rings is the ring of endomorphisms over a ring $R$.
\item For any group $G$, the group ring $R[G]$ is the set of formal sums of elements of $G$ with coefficients in $R$. 
\item For any non-empty set $M$ and a ring $R$, the set $R^M$ of all functions from $M$ to $R$ may be made a ring\, $(R^M,\,+,\,\cdot)$\, by setting for such functions $f$ and $g$
$$(f\!+\!g)(x) := f(x)+g(x), \,\,\, (fg)(x) := f(x)g(x)\,\,\, \forall x\in M.$$
This ring is the often denoted $\bigoplus_{M} R$.  For instance, if $M=\{1,2\}$, then $R^M\cong R\oplus R$.

\item If $R$ is commutative, the ring of fractions $S^{-1}R$ where $S$ is a multiplicative subset of $R$ not containing 0.
\item Let $S,T$ be subrings of $R$.  Then
$$\begin{pmatrix} S&R \\ 0&T \end{pmatrix}:=\Big\lbrace \begin{pmatrix} s&r \\ 0&t \end{pmatrix}\mid r\in R, s\in S, t\in T \Big\rbrace$$
with the usual matrix addition and multiplication is a ring.
\end{enumerate}
%%%%%
%%%%%
\end{document}
