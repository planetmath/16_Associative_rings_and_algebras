\documentclass[12pt]{article}
\usepackage{pmmeta}
\pmcanonicalname{AnnihilatorIsAnIdeal}
\pmcreated{2013-03-22 12:50:27}
\pmmodified{2013-03-22 12:50:27}
\pmowner{yark}{2760}
\pmmodifier{yark}{2760}
\pmtitle{annihilator is an ideal}
\pmrecord{10}{33167}
\pmprivacy{1}
\pmauthor{yark}{2760}
\pmtype{Theorem}
\pmcomment{trigger rebuild}
\pmclassification{msc}{16D10}
\pmclassification{msc}{16D25}

\endmetadata

\usepackage{amssymb}
\usepackage{amsmath}
\usepackage{amsfonts}

\begin{document}
\PMlinkescapeword{addition}
\PMlinkescapeword{equivalent}
\PMlinkescapeword{multiplication}
\PMlinkescapeword{right}

The right annihilator of a right $R$-module $M_R$ in $R$ is an ideal.

{\bf Proof:}\\
By the distributive law for modules, it is easy to see that $\operatorname{r.ann}(M_R)$ is closed under addition and right multiplication.
Now take $x \in \operatorname{r.ann}(M_R)$ and $r \in R$.

Take any $m \in M_R$.  Then $mr \in M_R$, but then $(mr)x = 0$ since $x \in \operatorname{r.ann}(M_R)$.  So $m(rx)=0$ and $rx \in \operatorname{r.ann}(M_R)$.

An equivalent result holds for left annihilators.
%%%%%
%%%%%
\end{document}
