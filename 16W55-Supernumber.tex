\documentclass[12pt]{article}
\usepackage{pmmeta}
\pmcanonicalname{Supernumber}
\pmcreated{2013-03-22 13:03:27}
\pmmodified{2013-03-22 13:03:27}
\pmowner{mhale}{572}
\pmmodifier{mhale}{572}
\pmtitle{supernumber}
\pmrecord{12}{33463}
\pmprivacy{1}
\pmauthor{mhale}{572}
\pmtype{Definition}
\pmcomment{trigger rebuild}
\pmclassification{msc}{16W55}
\pmrelated{SuperAlgebra}
\pmdefines{body}
\pmdefines{soul}

\endmetadata

\usepackage{amssymb}
\usepackage{amsmath}
\usepackage{amsfonts}
\usepackage{amsthm}

% used for TeXing text within eps files
%\usepackage{psfrag}
% need this for including graphics (\includegraphics)
%\usepackage{graphicx}
% making logically defined graphics
%%%\usepackage{xypic}

% my maths package

\newcommand*{\Nset}{\mathbb{N}}
\newcommand*{\Zset}{\mathbb{Z}}
\newcommand*{\Qset}{\mathbb{Q}}
\newcommand*{\Rset}{\mathbb{R}}
\newcommand*{\Cset}{\mathbb{C}}
\newcommand*{\Hset}{\mathbb{H}}
\newcommand*{\Oset}{\mathbb{O}}
\newcommand*{\Bset}{\mathbb{B}}
\newcommand*{\Kset}{\mathbb{K}}
\newcommand*{\Sset}{\mathbb{S}}
\newcommand*{\Tset}{\mathbb{T}}
\newcommand*{\GLgrp}{\mathrm{GL}}
\newcommand*{\SLgrp}{\mathrm{SL}}
\newcommand*{\Ogrp}{\mathrm{O}}
\newcommand*{\SOgrp}{\mathrm{SO}}
\newcommand*{\Ugrp}{\mathrm{U}}
\newcommand*{\SUgrp}{\mathrm{SU}}
\newcommand*{\e}{\mathop{\mathrm{e}}\nolimits}
\newcommand*{\im}{\mathord{\mathrm{i}}}
\newcommand*{\identity}{\mathord{\mathrm{1\!\!\!\:I}}}
\newcommand*{\tr}{\mathop{\mathrm{tr}}}
\newcommand*{\Tr}{\mathop{\mathrm{Tr}}}
\renewcommand*{\d}{\mathrm{d}}
\newcommand*{\deriv}[2]{\frac{\d #1}{\d #2}}
\newcommand*{\pderiv}[2]{\frac{\partial #1}{\partial #2}}
\newcommand*{\fderiv}[2]{\frac{\delta #1}{\delta #2}}

% my noncommutative geometry package

\newcommand*{\algebra}[1][A]{\mathord{\mathcal{#1}}}
\newcommand*{\hilbert}[1][H]{\mathord{\mathcal{#1}}}
\newcommand*{\hilbmod}[1][E]{\mathord{\mathcal{#1}}}
\newcommand*{\Matrix}[2]{\mathord{\mathrm{M}_{#1}(#2)}}
\newcommand*{\dixmier}{\mathop{\mathrm{Tr}_\omega}}
\newcommand*{\Res}{\mathop{\mathrm{Res}}}
\newcommand*{\Wres}{\mathop{\mathrm{Wres}}}
\newcommand*{\Aut}{\mathop{\mathrm{Aut}}\nolimits}
\newcommand*{\Inn}{\mathop{\mathrm{Inn}}\nolimits}
\newcommand*{\Out}{\mathop{\mathrm{Out}}\nolimits}
\newcommand*{\Diff}{\mathop{\mathrm{Diff}}\nolimits}
\newcommand*{\Ker}{\mathop{\mathrm{Ker}}\nolimits}
\newcommand*{\Coker}{\mathop{\mathrm{Coker}}\nolimits}
\newcommand*{\Img}{\mathop{\mathrm{Im}}\nolimits}
\newcommand*{\End}{\mathop{\mathrm{End}}\nolimits}
\newcommand*{\spin}{\mathop{\mathrm{spin}}\nolimits}
\newcommand*{\Ind}{\mathop{\mathrm{Ind}}\nolimits}
\newcommand*{\KK}{\mathit{KK}}
\newcommand*{\HH}{\mathit{HH}}
\newcommand*{\HC}{\mathit{HC}}
\newcommand*{\ch}{\mathop{\mathrm{ch}}\nolimits}

% my category theory package

\newcommand*{\mathcat}[1]{\mathord{\mathbf{#1}}}
\newcommand*{\id}{\mathrm{id}}
\newcommand*{\op}{\mathrm{op}}
\newcommand*{\boxprod}{\mathbin{\square}}

% my environments

\newtheoremstyle{inlinedefn}{}{0pt}{}{}{\bfseries}{.}{0.5em}{}
\theoremstyle{inlinedefn}
\newtheorem{definition}{Definition}

\newtheoremstyle{break}{\baselineskip}{\baselineskip}{\itshape}{}{\bfseries}{}{\newline}{}
\theoremstyle{break}
\newtheorem{example}{Example}

% misc commands

\newcommand*{\defn}[1]{\textbf{#1}}
\renewcommand*{\bar}[1]{\overline{#1}}
\begin{document}
Supernumbers are the generalisation of complex numbers to a commutative superalgebra of commuting and anticommuting ``numbers''.
They are primarily used in the description of \PMlinkescapetext{fermionic fields} in \PMlinkescapetext{quantum field theory}.

Let $\Lambda_N$ be the Grassmann algebra generated by $\theta^i$, $i = 1 \ldots N$,
such that $\theta^i\theta^j = -\theta^j\theta^i$ and $(\theta^i)^2 = 0$.
Denote by $\Lambda_\infty$, the Grassmann algebra of an infinite number of generators $\theta^i$.
A \defn{supernumber} is an element of $\Lambda_N$ or $\Lambda_\infty$.

Any supernumber $z$ can be expressed uniquely in the form
\[
z = z_0 + z_i \theta^i + \frac{1}{2} z_{ij} \theta^i\theta^j + \ldots
+ \frac{1}{n!} z_{i_1 \ldots i_n} \theta^{i_1} \ldots \theta^{i_n} + \ldots,
\]
where the coefficients $z_{i_1 \ldots i_n} \in \Cset$ are antisymmetric in their indices.

\section{Body and soul}

The \defn{body} of a supernumber $z$ is defined as $z_\mathrm{B} = z_0$,
and its \defn{soul} is defined as $z_\mathrm{S} = z-z_\mathrm{B}$.
If $z_\mathrm{B} \neq 0$ then $z$ has an inverse given by
\[
z^{-1} = \frac{1}{z_\mathrm{B}} \sum_{k=0}^\infty \left(-\frac{z_\mathrm{S}}{z_\mathrm{B}}\right)^k.
\]

\section{Odd and even}

A supernumber can be decomposed into the even and odd parts:
\begin{eqnarray*}
z_\mathrm{even} & = & z_0 + \frac{1}{2} z_{ij} \theta^i\theta^j + \ldots
+ \frac{1}{(2n)!} z_{i_1 \ldots i_{2n}} \theta^{i_1} \ldots \theta^{i_{2n}} + \ldots, \\
z_\mathrm{odd} & = & z_i \theta^i + \frac{1}{6} z_{ijk} \theta^i\theta^j\theta^k + \ldots
+ \frac{1}{(2n+1)!} z_{i_1 \ldots i_{2n+1}} \theta^{i_1} \ldots \theta^{i_{2n+1}} + \ldots.
\end{eqnarray*}
Even supernumbers commute with each other and are called \defn{c-numbers},
while odd supernumbers anticommute with each other and are called \defn{a-numbers}.
Note, the product of two c-numbers is even,
the product of a c-number and an a-number is odd,
and the product of two a-numbers is even.
The superalgebra $\Lambda_N$ has the vector space decomposition
$\Lambda_N = \Cset_c \oplus \Cset_a$,
where $\Cset_c$ is the space of c-numbers,
and $\Cset_a$ is the space of a-numbers.

\section{Conjugation and involution}

There are two ways, one can define a complex conjugation for supernumbers.
The first is to define a linear conjugation in complete analogy with complex numbers:
\[
\bar{(z_1 z_2)} = \bar{z_1} \;\bar{z_2}.
\]
The second way is to define an anti-linear involution:
\[
(z_1 z_2)^* = z_2^* z_1^*.
\]
The \PMlinkescapetext{difference} comes down to whether the product of two real odd supernumbers is real or imaginary.
%%%%%
%%%%%
\end{document}
