\documentclass[12pt]{article}
\usepackage{pmmeta}
\pmcanonicalname{RingOfEndomorphisms}
\pmcreated{2013-03-22 14:04:26}
\pmmodified{2013-03-22 14:04:26}
\pmowner{mclase}{549}
\pmmodifier{mclase}{549}
\pmtitle{ring of endomorphisms}
\pmrecord{8}{35432}
\pmprivacy{1}
\pmauthor{mclase}{549}
\pmtype{Definition}
\pmcomment{trigger rebuild}
\pmclassification{msc}{16S20}
\pmclassification{msc}{16W20}
\pmsynonym{endomorphism ring}{RingOfEndomorphisms}

% this is the default PlanetMath preamble.  as your knowledge
% of TeX increases, you will probably want to edit this, but
% it should be fine as is for beginners.

% almost certainly you want these
\usepackage{amssymb}
\usepackage{amsmath}
\usepackage{amsfonts}

% for neatly defining theorems and propositions
\usepackage{amsthm}

% used for TeXing text within eps files
%\usepackage{psfrag}
% need this for including graphics (\includegraphics)
%\usepackage{graphicx}
% making logically defined graphics
%%%\usepackage{xypic}

% there are many more packages, add them here as you need them

% define commands here

\newtheorem*{lemma}{Lemma}
\newcommand{\End}{\operatorname{End}}
\newcommand{\isom}{\cong}
\begin{document}
\PMlinkescapeword{lemma}
\PMlinkescapeword{order}
\PMlinkescapeword{maps}
\PMlinkescapeword{natural}
\PMlinkescapeword{notation}

Let $R$ be a ring and let $M$ be a right $R$-module.

An \emph{endomorphism} of $M$ is a $R$-module homomorphism from $M$ to itself.
We shall write endomorphisms on the left,
so that $f : M \to M$ maps $x \mapsto f(x)$.
If $f,g : M \to M$ are two endomorphisms, we can add them:
$$ f+g : x \mapsto f(x) + g(x) $$
and multiply them
$$ fg : x \mapsto f(g(x)) $$
With these operations, the set of endomorphisms of $M$ becomes a ring, which we call
the \emph{ring of endomorphisms of $M$}, written $\End_R(M)$.

Instead of writing endomorphisms as functions, it is often convenient
to write them multiplicatively: we simply write the application of the
endomorphism $f$ as $x \mapsto fx$.  Then the fact that each $f$ is an $R$-module homomorphism can be expressed as:
$$f(xr) = (fx)r$$
for all $x \in M$ and $r \in R$ and $f \in \End_R(M)$.
With this notation, it is clear that $M$ becomes an $\End_R(M)$-$R$-bimodule.

Now, let $N$ be a left $R$-module.  We can construct the ring $\End_R(N)$ in the same way.
There is a complication, however, if we still think of endomorphism as functions written
on the left.  In order to make $M$ into a bimodule, we need to define an action of
$\End_R(N)$ on the right of $N$: say
$$ x\cdot f = f(x) $$
But then we have a problem with the multiplication:
$$ x \cdot fg = fg(x) = f(g(x)) $$
but $$(x \cdot f) \cdot g = f(x) \cdot g = g(f(x))!$$
In order to make this work, we need to reverse the order of composition
when we define multiplication in the ring $\End_R(N)$ when it acts on the right.

There are essentially two different ways to go from here.  One is to define the multiplication
in $\End_R(N)$ the other way, which is most natural if we write the endomorphisms as functions
on the right.  This is the approach taken in many older books.

The other is to leave the
multiplication in $\End_R(N)$ the way it is, but to use the opposite ring to define the bimodule.
This is the approach that is generally taken in more recent works.  Using this approach,
we conclude that $N$ is a $R$-$\End_R(N)^{\text{op}}$-bimodule.   We will adopt this convention
for the lemma below.

Considering $R$ as a right and a left module over itself,
we can construct the two endomorphism rings $\End_R(R_R)$ and $\End_R({}_RR)$.

\begin{lemma}
Let $R$ be a ring with an identity element.
Then $R \isom \End_R(R_R)$ and $R \isom \End_R({}_RR)^{\text{op}}$.
\end{lemma}

\begin{proof}
Define $\rho_r \in \End_R({}_RR)$ by $x \mapsto xr$.

A calculation shows that $\rho_{rs} = \rho_s \rho_r$ (functions written on the left)
from which it is easily seen that the map $\theta : r \mapsto \rho_r$ is a ring
homomorphism from $R$ to $\End_R({}_RR)^{\text{op}}$.

We must show that this is an isomorphism.

If $\rho_r = 0$, then $r = 1r = \rho_r(1) = 0$.  So $\theta$ is injective.

Let $f$ be an arbitrary element of $\End_R({}_RR)$, and let $r = f(1)$.
Then for any $x \in R$, $f(x) = f(x1) = xf(1) = xr = \rho_r(x)$, so
$f = \rho_r = \theta(r)$.

The proof of the other isomorphism is similar.
\end{proof}
%%%%%
%%%%%
\end{document}
