\documentclass[12pt]{article}
\usepackage{pmmeta}
\pmcanonicalname{ExamplesOfModules}
\pmcreated{2013-03-22 14:36:28}
\pmmodified{2013-03-22 14:36:28}
\pmowner{mathcam}{2727}
\pmmodifier{mathcam}{2727}
\pmtitle{examples of modules}
\pmrecord{16}{36180}
\pmprivacy{1}
\pmauthor{mathcam}{2727}
\pmtype{Example}
\pmcomment{trigger rebuild}
\pmclassification{msc}{16-00}
\pmclassification{msc}{20-00}
\pmclassification{msc}{13-00}
%\pmkeywords{modules}
%\pmkeywords{rings}
%\pmkeywords{identity element}
%\pmkeywords{Yetter-Drinfel'd module}
\pmrelated{QuantumDouble}
\pmrelated{Module}

% this is the default PlanetMath preamble.  as your knowledge
% of TeX increases, you will probably want to edit this, but
% it should be fine as is for beginners.

% almost certainly you want these
\usepackage{amssymb}
\usepackage{amsmath}
\usepackage{amsfonts}

% used for TeXing text within eps files
%\usepackage{psfrag}
% need this for including graphics (\includegraphics)
%\usepackage{graphicx}
% for neatly defining theorems and propositions
%\usepackage{amsthm}
% making logically defined graphics
%%%\usepackage{xypic}

% there are many more packages, add them here as you need them

% define commands here
\begin{document}
This entry is a \PMlinkescapetext{collection} of examples of modules over rings.  Unless otherwise specified in the example, $M$ will be a module over a ring $R$.

\begin{itemize}
\item Any abelian group is a module over the ring of integers, with action defined by $n\cdot g$ for $g\in G$ given by $n\cdot g=\sum_{i=1}^n g$.
\item If $R$ is a subring of a ring $S$, then $S$ is an $R$-module, with action given by multiplication in $S$.
\item If $R$ is any ring, then any (left) ideal $I$ of $R$ is a (left) $R$-module, with action given by the multiplication in $R$.
\item Let $R=\mathbb{Z}$ and let $E = \{2k \mid k \in \mathbb{Z}\}$. Then $E$ is a module over the ring $\mathbb{Z}$ of integers. Further, define the sets $B = E \times E$ and $C = E \times \{0\}$ and $D = \{0\} \times  E$. Then $B$, $C$, and $D$ are modules over $\mathbb{Z} \times  \mathbb{Z}$, with action given by $a \cdot x = 
(a \cdot x_1, a \cdot x_2)$ if $x = (x_1,x_2)$ even if the product is redefined as $a \cdot x_1 = 0$ and $a \cdot x_2 = 0$, but now the identity element is $(1,1)$. However by our new product definition $a \cdot x = (a \cdot x_1, a \cdot x_2) = (0,0)$ even if $a = (1,1)$, the ring identity element originally  In the more general definition of module which does not require an identity element $\bf{1}$ in the ring and does not require ${\bf 1} \cdot m = m$ for all $m \in M$, we observe that ${\bf 1} \cdot m \neq m$  in this example just constructed.  (one of the purposes of this comment is to show that all modules need not be unital ones).
\item \PMlinkname{Yetter-Drinfel'd module.}{QuantumDouble}
\end{itemize}
%%%%%
%%%%%
\end{document}
