\documentclass[12pt]{article}
\usepackage{pmmeta}
\pmcanonicalname{IABIsInvertibleIfAndOnlyIfIBAIsInvertible}
\pmcreated{2013-03-22 14:44:43}
\pmmodified{2013-03-22 14:44:43}
\pmowner{asteroid}{17536}
\pmmodifier{asteroid}{17536}
\pmtitle{I-AB is invertible if and only if I-BA is invertible}
\pmrecord{16}{36383}
\pmprivacy{1}
\pmauthor{asteroid}{17536}
\pmtype{Theorem}
\pmcomment{trigger rebuild}
\pmclassification{msc}{16B99}
\pmclassification{msc}{15A04}
\pmclassification{msc}{47A10}
\pmrelated{TechniquesInMathematicalProofs}

\endmetadata

% this is the default PlanetMath preamble.  as your knowledge
% of TeX increases, you will probably want to edit this, but
% it should be fine as is for beginners.

% almost certainly you want these
\usepackage{amssymb}
\usepackage{amsmath}
\usepackage{amsfonts}
\usepackage{amsthm}

\usepackage{mathrsfs}

% used for TeXing text within eps files
%\usepackage{psfrag}
% need this for including graphics (\includegraphics)
%\usepackage{graphicx}
% for neatly defining theorems and propositions
%
% making logically defined graphics
%%%\usepackage{xypic}

% there are many more packages, add them here as you need them

% define commands here

\newcommand{\sR}[0]{\mathbb{R}}
\newcommand{\sC}[0]{\mathbb{C}}
\newcommand{\sN}[0]{\mathbb{N}}
\newcommand{\sZ}[0]{\mathbb{Z}}

 \usepackage{bbm}
 \newcommand{\Z}{\mathbbmss{Z}}
 \newcommand{\C}{\mathbbmss{C}}
 \newcommand{\R}{\mathbbmss{R}}
 \newcommand{\Q}{\mathbbmss{Q}}



\newcommand*{\norm}[1]{\lVert #1 \rVert}
\newcommand*{\abs}[1]{| #1 |}



\newtheorem{thm}{Theorem}
\newtheorem{defn}{Definition}
\newtheorem{prop}{Proposition}
\newtheorem{lemma}{Lemma}
\newtheorem{cor}{Corollary}
\begin{document}
\PMlinkescapeword{invertible}
\PMlinkescapeword{inverse}

In this entry $A$ and $B$ are endomorphisms of a vector space $V$. If $V$ is finite dimensional, we may choose a basis and regard $A$ and $B$ as square matrices of equal dimension.

{\bf Theorem -} Let $A$ and $B$ be endomorphisms of a vector space $V$. We have that
\begin{enumerate}
\item $I-AB$ is \PMlinkname{invertible}{LinearIsomorphism} if and only if $I-BA$ is invertible, and moreover
\item $I-AB$ is injective if and only if $I-BA$ is injective.
\end{enumerate}

{\bf Proof :}
\begin{enumerate}
\item Suppose that $I-AB$ is invertible. We shall prove that $B(I-AB)^{-1}A +I$ is the inverse of $I-BA$.
 In fact
\begin{eqnarray*}
\Big(I-BA\Big)\Big(B(I-AB)^{-1}A +I\Big) &=& B(I-AB)^{-1}A + I - BAB(I-AB)^{-1}A - BA\\
&=& B \Big((I-AB)^{-1} - AB(I-AB)^{-1}\Big)A + I - BA\\
&=& B \Big((I-AB)(I-AB)^{-1}\Big)A + I -BA\\
&=& BA + I - BA\\
&=& I
\end{eqnarray*}

A similar computation shows that $\Big(B(I-AB)^{-1}A +I\Big)\Big(I-BA\Big) = I$, i.e. $I - BA$ is invertible.

Exchanging the roles of $A$ and $B$ we can prove the "if" part. So $I-AB$ is invertible if and only if $I - BA$ is invertible.

\item Let us first 
recall that a linear map between vector spaces is
invertible if and only if its kernel $\operatorname{ker}$
is the zero vector (see \PMlinkname{this page}{KernelOfALinearTransformation}).

Suppose $I - AB$ is not injective, i.e. there exists $u \neq 0$ such that $(I-AB)u=0$. Then
\begin{displaymath}
(I-BA)Bu = B(I-AB)u = 0
\end{displaymath}

i.e. $Bu \in \operatorname{ker}(I-BA)$. Notice that $Bu \neq 0$ because $u = ABu$ (by definition of $u$), so 
$I-BA$ is also not injective.

Similarly, if $I-BA$ is not injective then $I - AB$ is not injective. $\square$

\end{enumerate}

{\bf Remark -} It is known that for finite dimensional vector spaces a linear endomorphism is invertible if and only if it is injective. This does not remain true for infinite dimensional spaces, hence 1 and 2 are two different statements.

\subsection{Comments}
The result stated in 1 can be proven in a more general context --- If $A$ and $B$ are elements of a ring with unity, then $I-AB$ is invertible if and only if $I-BA$ is invertible. See the entry on techniques in mathematical proofs, in which this result is proven using several different techniques.

This entry is based on \PMlinkexternal{this discussion on PM}{http://planetmath.org/?op=getmsg&id=5088}.
%%%%%
%%%%%
\end{document}
