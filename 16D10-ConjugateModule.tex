\documentclass[12pt]{article}
\usepackage{pmmeta}
\pmcanonicalname{ConjugateModule}
\pmcreated{2013-03-22 11:49:47}
\pmmodified{2013-03-22 11:49:47}
\pmowner{antizeus}{11}
\pmmodifier{antizeus}{11}
\pmtitle{conjugate module}
\pmrecord{9}{30377}
\pmprivacy{1}
\pmauthor{antizeus}{11}
\pmtype{Definition}
\pmcomment{trigger rebuild}
\pmclassification{msc}{16D10}
\pmclassification{msc}{41A45}

\endmetadata

\usepackage{amssymb}
\usepackage{amsmath}
\usepackage{amsfonts}
\usepackage{graphicx}
%%%%\usepackage{xypic}
\begin{document}
If $M$ is a right module over a ring $R$,
and $\alpha$ is an endomorphism of $R$,
we define the {\it conjugate module} $M^\alpha$ 
to be the right $R$-module
whose underlying set is $\{ m^\alpha \mid m \in M \}$,
with abelian group structure identical to that of $M$
(i.e. $(m-n)^\alpha = m^\alpha - n^\alpha$),
and scalar multiplication given by
${m^\alpha} \cdot r = (m \cdot \alpha(r))^\alpha$
for all $m$ in $M$ and $r$ in $R$. 

In other words, if $\phi: R \to {\rm End}_{\Bbb Z}(M)$
is the ring homomorphism that describes 
the right module action of $R$ upon $M$,
then $\phi \alpha$ describes
the right module action of $R$ upon $M^\alpha$.


If $N$ is a left $R$-module, we define ${^\alpha N}$ similarly,
with $r \cdot {^\alpha n} = {^\alpha(\alpha(r) \cdot n)}$.
%%%%%
%%%%%
%%%%%
%%%%%
\end{document}
