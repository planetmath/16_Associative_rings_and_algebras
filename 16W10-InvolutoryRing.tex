\documentclass[12pt]{article}
\usepackage{pmmeta}
\pmcanonicalname{InvolutoryRing}
\pmcreated{2013-03-22 15:41:01}
\pmmodified{2013-03-22 15:41:01}
\pmowner{CWoo}{3771}
\pmmodifier{CWoo}{3771}
\pmtitle{involutory ring}
\pmrecord{32}{37625}
\pmprivacy{1}
\pmauthor{CWoo}{3771}
\pmtype{Definition}
\pmcomment{trigger rebuild}
\pmclassification{msc}{16W10}
\pmsynonym{ring admitting an involution}{InvolutoryRing}
\pmsynonym{involutary ring}{InvolutoryRing}
\pmsynonym{involutive ring}{InvolutoryRing}
\pmsynonym{ring with involution}{InvolutoryRing}
\pmsynonym{Hermitian element}{InvolutoryRing}
\pmsynonym{symmetric element}{InvolutoryRing}
\pmsynonym{self-adjoint}{InvolutoryRing}
\pmsynonym{adjoint}{InvolutoryRing}
\pmsynonym{projection}{InvolutoryRing}
\pmsynonym{involutive ring}{InvolutoryRing}
\pmrelated{HollowMatrixRings}
\pmdefines{involution}
\pmdefines{adjoint element}
\pmdefines{self-adjoint element}
\pmdefines{projection element}
\pmdefines{norm element}
\pmdefines{trace element}
\pmdefines{skew symmetric element}
\pmdefines{*-homomorphism}
\pmdefines{normal element}
\pmdefines{unitary element}

\usepackage{amssymb,amscd}
\usepackage{amsmath}
\usepackage{amsfonts}
%\usepackage{bbm}

% used for TeXing text within eps files
%\usepackage{psfrag}
% need this for including graphics (\includegraphics)
%\usepackage{graphicx}
% for neatly defining theorems and propositions
%\usepackage{amsthm}
% making logically defined graphics
%%%\usepackage{xypic}

% define commands here
\begin{document}
\subsubsection*{General Definition of a Ring with Involution}
Let $R$ be a ring.  An \emph{\PMlinkescapetext{involution}} $*$ on $R$ is an anti-endomorphism whose square is the identity map.  In other words, for $a,b\in R$:

\begin{enumerate}
\item $(a+b)^*=a^*+b^*$,
\item $(ab)^*=b^*a^*$,
\item $a^{**}=a$
\end{enumerate}

A ring admitting an involution is called an \emph{involutory ring}.  $a^*$ is called the \emph{adjoint} of $a$.  By (3), $a$ is the adjoint of $a^*$, so that every element of $R$ is an adjoint.  

\textbf{Remark}.  Note that the traditional definition of an \PMlinkname{involution}{Involution} on a vector space is different from the one given here.  Clearly, $*$ is bijective, so that it is an anti-automorphism.  If $*$ is the identity on $R$, then $R$ is commutative.

\textbf{Examples}.  Involutory rings occur most often in rings of endomorphisms over a module.  When $V$ is a finite dimensional vector space over a field $k$ with a given basis $\boldsymbol{b}$, any linear transformation over $T$ (to itself) can be represented by a square matrix $M$ over $k$ via $\boldsymbol{b}$.  The map taking $M$ to its transpose $M^T$ is an involution.  If $k$ is $\mathbb{C}$, then the map taking $M$ to its conjugate transpose $\overline{M}^T$ is also an involution.  In general, the composition of an isomorphism and an involution is an involution, and the composition of two involutions is an isomorphism.

\subsubsection*{*-Homomorphisms}  Let $R$ and $S$ be involutory rings with involutions $*_R$ and $*_S$.  A \emph{*-homomorphism} $\phi:R\to S$ is a ring homomorphism which respects involutions.  More precisely,

$$\phi(a^{*_R})=\phi(a)^{*_S},\quad\mbox{ for any }a\in R.$$

By abuse of notation, if we use $*$ to denote both $*_R$ and $*_S$, then we see that any *-homomorphism $\phi$ commutes with $*$: $\phi*=*\phi$.

\subsubsection*{Special Elements}  An element $a\in R$ such that $a=a^*$ is called a \emph{self-adjoint}.  A ring with involution is usually associated with a ring of square matrices over a field, as such, a self-adjoint element is sometimes called a \emph{Hermitian element}, or a \emph{symmetric element}.  For example, for any element $a\in R$, 

\begin{enumerate}
\item $aa^*$ and $a^*a$ are both self-adjoint, the first of which is called the \emph{norm} of $a$.  A \emph{norm element} $b$ is simply an element expressible in the form $aa^*$ for some $a\in R$, and we write $b=\operatorname{n}(a)$.  If $aa^*=a^*a$, then $a$ is called a \emph{normal element}.  If $a^*$ is the multiplicative inverse of $a$, then $a$ is a \emph{unitary element}.  If $a$ is unitary, then it is normal.
\item With respect to addition, we can also form self-adjoint elements $a+a^*=a^*+a$, called the \emph{trace} of $a$, for any $a\in R$.  A \emph{trace element} $b$ is an element expressible as $a+a^*$ for some $a\in R$, and written $b=\operatorname{tr}(a)$.
\end{enumerate}

Let $S$ be a subset of $R$, write $S^*:=\lbrace a^*\mid a\in S\rbrace$.  Then $S$ is said to be \emph{self-adjoint} if $S=S^*$.  

A self-adjoint that is also an idempotent in $R$ is called a \emph{projection}.  If $e$ and $f$ are two projections in $R$ such that $eR=fR$ (principal ideals generated by $e$ and $f$ are equal), then $e=f$.  For if $ea=ff=f$ for some $a\in R$, then $f=ea=eea=ef$.  Similarly, $e=fe$.  Therefore, $e=e^*=(fe)^*=e^*f^*=ef=f$.

If the characteristic of $R$ is not 2, we also have a companion concept to self-adjointness, that of skew symmetry.  An element $a$ in $R$ is skew symmetric if $a=-a^*$.  Again, the name of this is borrowed from linear algebra.
%%%%%
%%%%%
\end{document}
