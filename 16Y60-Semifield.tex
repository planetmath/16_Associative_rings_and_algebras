\documentclass[12pt]{article}
\usepackage{pmmeta}
\pmcanonicalname{Semifield}
\pmcreated{2013-03-22 15:45:46}
\pmmodified{2013-03-22 15:45:46}
\pmowner{CWoo}{3771}
\pmmodifier{CWoo}{3771}
\pmtitle{semifield}
\pmrecord{7}{37718}
\pmprivacy{1}
\pmauthor{CWoo}{3771}
\pmtype{Definition}
\pmcomment{trigger rebuild}
\pmclassification{msc}{16Y60}
\pmclassification{msc}{12K10}
\pmrelated{NonAssociativeAlgebra}

% this is the default PlanetMath preamble.  as your knowledge
% of TeX increases, you will probably want to edit this, but
% it should be fine as is for beginners.

% almost certainly you want these
\usepackage{amssymb}
\usepackage{amsmath}
\usepackage{amsfonts}

% used for TeXing text within eps files
%\usepackage{psfrag}
% need this for including graphics (\includegraphics)
%\usepackage{graphicx}
% for neatly defining theorems and propositions
 \usepackage{amsthm}
% making logically defined graphics
%%%\usepackage{xypic}

% there are many more packages, add them here as you need them

% define commands here

\theoremstyle{definition}
\newtheorem*{thmplain}{Theorem}
\begin{document}
There are different definitions of {\em semifield}.\, We give three such which are not \PMlinkname{equivalent}{Biconditional}.

Let $K$ be a set with two binary operations ``$+$'' and ``$\cdot$''.
\begin{itemize}
\item Semifield\, $(K,\,+,\,\cdot)$\, is a semiring where all non-zero elements have a multiplicative inverse.
\item Semifield\, is the algebraic system\, $(K,\,+,\,\cdot)$,\, where\, $(K,\,+)$\, is a group (identity $:= 0$),\, the multiplication ``$\cdot$'' distributes over the addition ``$+$'',\, $K$ \PMlinkescapetext{contains} the multiplicative identity\, $:= 1$\, and all equations\, $ax = b$\, and\, $ya = b$\, with\, $a \ne 0$\, have solutions $x$, $y$ in $K$.
\item Semifield\, $(K,\,+,\,\cdot)$\, satisfies all postulates of field except the associativity of the multiplication ``$\cdot$''.
\end{itemize}
%%%%%
%%%%%
\end{document}
