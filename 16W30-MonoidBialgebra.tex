\documentclass[12pt]{article}
\usepackage{pmmeta}
\pmcanonicalname{MonoidBialgebra}
\pmcreated{2013-03-22 18:58:48}
\pmmodified{2013-03-22 18:58:48}
\pmowner{joking}{16130}
\pmmodifier{joking}{16130}
\pmtitle{monoid bialgebra}
\pmrecord{4}{41844}
\pmprivacy{1}
\pmauthor{joking}{16130}
\pmtype{Example}
\pmcomment{trigger rebuild}
\pmclassification{msc}{16W30}

\endmetadata

% this is the default PlanetMath preamble.  as your knowledge
% of TeX increases, you will probably want to edit this, but
% it should be fine as is for beginners.

% almost certainly you want these
\usepackage{amssymb}
\usepackage{amsmath}
\usepackage{amsfonts}

% used for TeXing text within eps files
%\usepackage{psfrag}
% need this for including graphics (\includegraphics)
%\usepackage{graphicx}
% for neatly defining theorems and propositions
%\usepackage{amsthm}
% making logically defined graphics
%%%\usepackage{xypic}

% there are many more packages, add them here as you need them

% define commands here

\begin{document}
Let $G$ be a monoid and $k$ a field. Consider the vector space $kG$ over $k$ with basis $G$. More precisely,
$$kG=\{f:G\to k\ |\ f(g)=0 \mbox{ for almost all }g\in G\}.$$
We identify $g\in G$ with a function $f_g:G\to k$ such that $f_g(g)=1$ and $f_g(h)=0$ for $h\neq g$. Thus, every element in $kG$ is of the form 
$$\sum_{g\in G}\lambda_g g,$$
for $\lambda_g\in k$. The vector space $kG$ can be turned into a $k$-algebra, if we define multiplication as follows:
$$g\cdot h=gh,$$
where on the right side we have a multiplication in the monoid $G$. This definition extends linearly to entire $kG$ and defines an algebra structure on $kG$, where neutral element of $G$ is the identity in $kG$.

Furthermore, we can turn $kG$ into a coalgebra as follows: comultiplication $\Delta:kG\to kG\otimes kG$ is defined by $\Delta(g)=g\otimes g$ and counit $\varepsilon:kG\to k$ is defined by $\varepsilon(g)=1$. One can easily check that this defines coalgebra structure on $kG$.

The vector space $kG$ is a bialgebra with with these algebra and coalgebra structures and it is called a \textit{monoid bialgebra}.
%%%%%
%%%%%
\end{document}
