\documentclass[12pt]{article}
\usepackage{pmmeta}
\pmcanonicalname{PropertiesOfTheOrdinaryQuiver}
\pmcreated{2013-03-22 19:17:44}
\pmmodified{2013-03-22 19:17:44}
\pmowner{joking}{16130}
\pmmodifier{joking}{16130}
\pmtitle{properties of the ordinary quiver}
\pmrecord{4}{42230}
\pmprivacy{1}
\pmauthor{joking}{16130}
\pmtype{Theorem}
\pmcomment{trigger rebuild}
\pmclassification{msc}{16S99}
\pmclassification{msc}{20C99}
\pmclassification{msc}{13B99}

\endmetadata

% this is the default PlanetMath preamble.  as your knowledge
% of TeX increases, you will probably want to edit this, but
% it should be fine as is for beginners.

% almost certainly you want these
\usepackage{amssymb}
\usepackage{amsmath}
\usepackage{amsfonts}

% used for TeXing text within eps files
%\usepackage{psfrag}
% need this for including graphics (\includegraphics)
%\usepackage{graphicx}
% for neatly defining theorems and propositions
%\usepackage{amsthm}
% making logically defined graphics
%%%\usepackage{xypic}

% there are many more packages, add them here as you need them

% define commands here

\begin{document}
Let $k$ be a field and $A$ be a finite-dimensional algebra over $k$. Denote by $Q_A$ \PMlinkname{the ordinary quiver}{OrdinaryQuiverOfAnAlgebra} of $A$.

\textbf{Theorem.} The following statements hold:
\begin{enumerate}
\item If $A$ is basic and connected, then $Q_A$ is a connected quiver.
\item If $Q$ is a finite quiver and $I$ is an \PMlinkname{admissible ideal}{AdmissibleIdealsBoundQuiverAndItsAlgebra} in $kQ$ and $A=kQ/I$, then $Q_A$ and $Q$ are isomorphic.
\item If $A$ is basic and connected, then $A$ is isomorphic to $kQ_A/I$ for some (not necessarily unique) \PMlinkname{admissible ideal}{AdmissibleIdealsBoundQuiverAndItsAlgebra} $I$.
\end{enumerate}

For proofs please see \cite[Chapter II.3]{ASS}.

\begin{thebibliography}{99}
\bibitem{ASS} I. Assem, D. Simson, A. Skowronski, \textit{Elements of the Representation Theory of Associative Algebras, vol 1.}, Cambridge University Press 2006, 2007
\end{thebibliography}

%%%%%
%%%%%
\end{document}
