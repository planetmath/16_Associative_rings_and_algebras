\documentclass[12pt]{article}
\usepackage{pmmeta}
\pmcanonicalname{BehaviorExistsUniquelyfiniteCase}
\pmcreated{2013-03-22 16:02:35}
\pmmodified{2013-03-22 16:02:35}
\pmowner{Wkbj79}{1863}
\pmmodifier{Wkbj79}{1863}
\pmtitle{behavior exists uniquely (finite case)}
\pmrecord{13}{38093}
\pmprivacy{1}
\pmauthor{Wkbj79}{1863}
\pmtype{Proof}
\pmcomment{trigger rebuild}
\pmclassification{msc}{16U99}
\pmclassification{msc}{13M05}
\pmclassification{msc}{13A99}

\usepackage{amssymb}
\usepackage{amsmath}
\usepackage{amsfonts}

\usepackage{psfrag}
\usepackage{graphicx}
\usepackage{amsthm}
%%\usepackage{xypic}
\begin{document}
\PMlinkescapeword{generator}
\PMlinkescapeword{generators}
\PMlinkescapeword{order}

The following is a proof that behavior exists uniquely for any finite cyclic ring $R$.

\begin{proof}
Let $n$ be the \PMlinkname{order}{OrderRing} of $R$ and $r$ be a \PMlinkname{generator}{Generator} of the additive group of $R$.  Then there exists $a \in \mathbb{Z}$ with $r^2=ar$.  Let $k=\gcd(a,n)$ and $b \in \mathbb{Z}$ with $a=bk$.  Since $\gcd(b,n)=1$, there exists $c \in \mathbb{Z}$ with $bc \equiv 1 \operatorname{mod} n$.  Since $\gcd(c,n)=1$, $cr$ is a generator of the additive group of $R$.  Since $(cr)^2=c^2r^2=c^2(ar)=c^2(bkr)=c(bc)(kr)=k(cr)$, it follows that $k$ is a behavior of $R$.  Thus, existence of behavior has been proven.

Let $g$ and $h$ be behaviors of $R$.  Then there exist generators $s$ and $t$ of the additive group of $R$ such that $s^2=gs$ and $t^2=ht$.  Since $t$ is a generator of the additive group of $R$, there exists $w \in \mathbb{Z}$ with $\gcd(w,n)=1$ such that $t=ws$.

Note that $(hw)s=h(ws)=ht=t^2=(ws)^2=w^2s^2=w^2(gs)=(gw^2)s$.  Thus, $gw^2 \equiv hw \operatorname{mod} n$.  Recall that $\gcd(w,n)=1$.  Therefore, $gw \equiv h \operatorname{mod} n$.  Since $g$ and $h$ are both positive divisors of $n$ and $\gcd(w,n)=1$, it follows that $g=\gcd(g,n)=\gcd(gw,n)=\gcd(h,n)=h$.  Thus, uniqueness of behavior has been proven.
\end{proof}

Note that it has also been shown that, if $R$ is a finite cyclic ring of order $n$, $r$ is a generator of the additive group of $R$, and $a \in \mathbb{Z}$ with $r^2=ar$, then the behavior of $R$ is $\gcd(a,n)$.
%%%%%
%%%%%
\end{document}
