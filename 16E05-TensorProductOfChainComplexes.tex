\documentclass[12pt]{article}
\usepackage{pmmeta}
\pmcanonicalname{TensorProductOfChainComplexes}
\pmcreated{2013-03-22 16:13:21}
\pmmodified{2013-03-22 16:13:21}
\pmowner{Mazzu}{14365}
\pmmodifier{Mazzu}{14365}
\pmtitle{tensor product of chain complexes}
\pmrecord{13}{38321}
\pmprivacy{1}
\pmauthor{Mazzu}{14365}
\pmtype{Definition}
\pmcomment{trigger rebuild}
\pmclassification{msc}{16E05}
\pmclassification{msc}{18G35}
%\pmkeywords{chain complex}
\pmdefines{tensor product of chain complexes}

\endmetadata

% this is the default PlanetMath preamble.  as your knowledge
% of TeX increases, you will probably want to edit this, but
% it should be fine as is for beginners.

% almost certainly you want these
\usepackage{amssymb}
\usepackage{amsmath}
\usepackage{amsfonts}

% used for TeXing text within eps files
%\usepackage{psfrag}
% need this for including graphics (\includegraphics)
%\usepackage{graphicx}
% for neatly defining theorems and propositions
%\usepackage{amsthm}
% making logically defined graphics
%%%\usepackage{xypic}

% there are many more packages, add them here as you need them

% define commands here

\begin{document}
\PMlinkescapeword{tensor product}

\newcommand{\cbra}[1]{\left( #1 \right)}
\newcommand{\qbra}[1]{\left[ #1 \right]}
\newcommand{\gbra}[1]{\left\{ #1 \right\}}
\newcommand{\abra}[1]{\left\langle #1 \right\rangle}

\newcommand{\pa}[1]{\partial_{#1}}
\newcommand{\pap}[1]{\partial_{#1} '}
\newcommand{\papp}[1]{\partial_{#1} ''}


Let $ C'=\gbra{C_n',\pap n}$ and $C''=\gbra{C_n'',\papp n}$ be two chain complexes of $R$-modules, where $R$ is a commutative ring with unity. Their \emph{tensor product} $C'\otimes_R C''=\gbra{(C'\otimes_R C'')_n,\pa n}$ is the chain complex defined by
$$ (C'\otimes_R C'')_n = \bigoplus_{i+j=n}(C_i'\otimes_R C_j''), $$
$$ \pa n(t'_i\otimes_R s''_j) = \pap i(t'_i)\otimes_R s''_j + (-1)^i\, t'_i\otimes_R \papp j(s''_j),\ \ \ \forall t'_i\in C_i',\ s''_j\in C_j'',\ (i+j=n),$$
where $C_i'\otimes_R C_j''$ denotes the \PMlinkname{tensor product}{TensorProduct} of $R$-modules $C_i'$ and $C_j''$.

Indeed, this defines a chain complex, because  for each $t'_i\otimes_R s''_j\in C_i'\otimes_R C_j''\subseteq (C'\otimes_R C'')_{i+j}$ we have
$$\pa{i+j-1} \pa {i+j}(t'_i\otimes_R s''_j) = \pa{i+j-1}\cbra{ \pap i(t'_i)\otimes_R s''_j + (-1)^i\, t'_i\otimes_R \papp j(s''_j) }= $$
$$ = (-1)^{i-1}\, \pap i(t'_i)\otimes_R \papp j(s''_j)+(-1)^i \pap i(t'_i)\otimes_R \papp j(s''_j)=0, $$
thus $C'\otimes_R C''$ is a chain complex.
%%%%%
%%%%%
\end{document}
