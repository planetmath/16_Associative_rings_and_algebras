\documentclass[12pt]{article}
\usepackage{pmmeta}
\pmcanonicalname{MaximalIdealIsPrimegeneralCase}
\pmcreated{2013-03-22 17:38:02}
\pmmodified{2013-03-22 17:38:02}
\pmowner{mclase}{549}
\pmmodifier{mclase}{549}
\pmtitle{maximal ideal is prime (general case)}
\pmrecord{8}{40055}
\pmprivacy{1}
\pmauthor{mclase}{549}
\pmtype{Theorem}
\pmcomment{trigger rebuild}
\pmclassification{msc}{16D25}
\pmrelated{MaximalIdealIsPrime}

% this is the default PlanetMath preamble.  as your knowledge
% of TeX increases, you will probably want to edit this, but
% it should be fine as is for beginners.

% almost certainly you want these
\usepackage{amssymb}
\usepackage{amsmath}
\usepackage{amsfonts}

% used for TeXing text within eps files
%\usepackage{psfrag}
% need this for including graphics (\includegraphics)
%\usepackage{graphicx}
% for neatly defining theorems and propositions
%\usepackage{amsthm}
% making logically defined graphics
%%%\usepackage{xypic}

% there are many more packages, add them here as you need them

% define commands here

\begin{document}
\PMlinkescapeword{prime}
\textbf{Theorem.} In a ring (not necessarily commutative) with unity, any maximal ideal is a prime ideal.

{\em Proof.}\, Let $\mathfrak{m}$ be a maximal ideal of such a ring $R$
and suppose $R$ has ideals $\mathfrak{a}$
and $\mathfrak{b}$ with $\mathfrak{a}\mathfrak{b} \subseteq \mathfrak{m}$,
but $\mathfrak{a} \nsubseteq \mathfrak{m}$.
Since $\mathfrak{m}$ is maximal, we must have $\mathfrak{a} + \mathfrak{m} = R$.
Then,
$$\mathfrak{b} = R\mathfrak{b} = (\mathfrak{a} + \mathfrak{m})\mathfrak{b}
= \mathfrak{a}\mathfrak{b} + \mathfrak{m}\mathfrak{b} \subseteq \mathfrak{m} + \mathfrak{m}
= \mathfrak{m}.$$
Thus, either $\mathfrak{a} \subseteq \mathfrak{m}$ or $\mathfrak{b} \subseteq \mathfrak{m}$.  This demonstrates that $\mathfrak{m}$ is prime.

Note that the condition that $R$ has an identity element is essential.  For otherwise, we may take $R$ to be a finite zero ring.  Such rings contain no proper prime ideals.  As long as the number of elements of $R$ is not prime, $R$ will have a non-zero maximal ideal. 


%%%%%
%%%%%
\end{document}
