\documentclass[12pt]{article}
\usepackage{pmmeta}
\pmcanonicalname{NestedIdealsInVonNeumannRegularRing}
\pmcreated{2013-03-22 14:48:24}
\pmmodified{2013-03-22 14:48:24}
\pmowner{CWoo}{3771}
\pmmodifier{CWoo}{3771}
\pmtitle{nested ideals in von Neumann regular ring}
\pmrecord{12}{36463}
\pmprivacy{1}
\pmauthor{CWoo}{3771}
\pmtype{Theorem}
\pmcomment{trigger rebuild}
\pmclassification{msc}{16E50}

\endmetadata

% this is the default PlanetMath preamble.  as your knowledge
% of TeX increases, you will probably want to edit this, but
% it should be fine as is for beginners.

% almost certainly you want these
\usepackage{amssymb}
\usepackage{amsmath}
\usepackage{amsfonts}

% used for TeXing text within eps files
%\usepackage{psfrag}
% need this for including graphics (\includegraphics)
%\usepackage{graphicx}
% for neatly defining theorems and propositions
 \usepackage{amsthm}
% making logically defined graphics
%%%\usepackage{xypic}

% there are many more packages, add them here as you need them

% define commands here
\theoremstyle{definition}
\newtheorem*{thmplain}{Theorem}
\begin{document}
\begin{thmplain}
\,\, Let $\mathfrak{a}$ be an ideal of the von Neumann regular ring $R$. \,Then $\mathfrak{a}$ itself is a von Neumann regular ring and any ideal $\mathfrak{b}$ of $\mathfrak{a}$ is likewise an ideal of $R$.
\end{thmplain}

\begin{proof} \,If \,$a\in \mathfrak{a}$, then \,$asa = a$\, for some \,$s\in R$. \,Setting \,$t = sas$\, we see that $t$ belongs to the ideal $\mathfrak{a}$ and \PMlinkescapetext{satisfies}
        $$ata = a(sas)a = (asa)sa = asa = a.$$
Secondly, we have to show that whenever \,$b\in\mathfrak{b} \subseteq \mathfrak{a}$\, and \,$r\in R$, then both $br$ and $rb$ lie in $\mathfrak{b}$. \,Now, \,$br\in \mathfrak{a}$\, because $\mathfrak{a}$ is an ideal of $R$. \,Thus there is an element $x$ in $\mathfrak{a}$ satisfying \,$brxbr = br$. \,Since $rxbr$ belongs to  $\mathfrak{a}$ and $\mathfrak{b}$ is assumed to be an ideal of $\mathfrak{a}$, we conclude that the product \,$b\cdot rxbr$\, must lie in $\mathfrak{b}$, i.e. \,$br\in\mathfrak{b}$. \,Similarly it can be shown that \,$rb\in\mathfrak{b}$.
\end{proof}

\begin{thebibliography}{9}
\bibitem{Burton} David M. Burton: {\em A first course in rings and ideals}. \,Addison-Wesley. \,Reading, Massachusetts (1970).
\end{thebibliography}
%%%%%
%%%%%
\end{document}
