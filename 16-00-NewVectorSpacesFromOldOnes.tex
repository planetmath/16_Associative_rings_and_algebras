\documentclass[12pt]{article}
\usepackage{pmmeta}
\pmcanonicalname{NewVectorSpacesFromOldOnes}
\pmcreated{2013-03-22 15:31:08}
\pmmodified{2013-03-22 15:31:08}
\pmowner{matte}{1858}
\pmmodifier{matte}{1858}
\pmtitle{new vector spaces from old ones}
\pmrecord{16}{37388}
\pmprivacy{1}
\pmauthor{matte}{1858}
\pmtype{Topic}
\pmcomment{trigger rebuild}
\pmclassification{msc}{16-00}
\pmclassification{msc}{13-00}
\pmclassification{msc}{20-00}
\pmclassification{msc}{15-00}

\endmetadata

% this is the default PlanetMath preamble.  as your knowledge
% of TeX increases, you will probably want to edit this, but
% it should be fine as is for beginners.

% almost certainly you want these
\usepackage{amssymb}
\usepackage{amsmath}
\usepackage{amsfonts}
\usepackage{amsthm}

\usepackage{mathrsfs}

% used for TeXing text within eps files
%\usepackage{psfrag}
% need this for including graphics (\includegraphics)
%\usepackage{graphicx}
% for neatly defining theorems and propositions
%
% making logically defined graphics
%%%\usepackage{xypic}

% there are many more packages, add them here as you need them

% define commands here

\newcommand{\sR}[0]{\mathbb{R}}
\newcommand{\sC}[0]{\mathbb{C}}
\newcommand{\sN}[0]{\mathbb{N}}
\newcommand{\sZ}[0]{\mathbb{Z}}

 \usepackage{bbm}
 \newcommand{\Z}{\mathbbmss{Z}}
 \newcommand{\C}{\mathbbmss{C}}
 \newcommand{\F}{\mathbbmss{F}}
 \newcommand{\R}{\mathbbmss{R}}
 \newcommand{\Q}{\mathbbmss{Q}}



\newcommand*{\norm}[1]{\lVert #1 \rVert}
\newcommand*{\abs}[1]{| #1 |}



\newtheorem{thm}{Theorem}
\newtheorem{defn}{Definition}
\newtheorem{prop}{Proposition}
\newtheorem{lemma}{Lemma}
\newtheorem{cor}{Corollary}
\begin{document}
This entry list methods that give new vector spaces from old ones.

\begin{enumerate}
\item Changing the field (complexification, etc.)
\item vector subspace
\item Quotient vector space
\item direct product of vectors spaces
\item Cartesian product of vector spaces
\item \PMlinkname{Tensor product of vector spaces}{TensorProductClassical}
\item The space of linear maps from one vector space to another, also denoted by $\operatorname{Hom}_k(V,W)$, or simply $\operatorname{Hom}(V,W)$, where $V$ and $W$ are vector spaces over the field $k$
\item The space of endomorphisms of a vector space.  Using the notation above, this is the space $\operatorname{Hom}_k(V,V)=\operatorname{End}(V)$
\item \PMlinkname{dual vector space}{DualSpace}, and bi-dual vector space.  Using the notation above, this is the space $\operatorname{Hom}(V,k)$, or simply $V^*$.
\item The annihilator of a subspace is a subspace of the dual vector space
\item Wedge product of vector spaces
\item A field $k$ is a vector space over itself.  Consider a set $B$ and the set $V$ of all functions from $B$ to $k$.  Then $V$ has a natural vector space structure.  If $B$ is finite, then $V$ can be viewed as a vector space having $B$ as a basis.
\end{enumerate}

\subsubsection*{Vector spaces involving a linear map}

Suppose $L\colon\thinspace V\to W$ is a linear map.

\begin{enumerate}
\item The kernel of $L$ is a subspace of $V$.
\item The image of $L$ is a subspace of $W$.
\item The cokernel of $L$ is a quotient space of $W$. 
\end{enumerate}

\subsubsection*{Topological vector spaces}

Suppose $V$ is topological vector space.
\begin{enumerate}
\item If $W$ is a subspace of $V$ then its closure $\overline{W}$ is also a subspace of $V$.
\item If $V$ is a metric vector space then its completion $\widetilde{V}$ is also a (metric) vector space.
\item The direct integral of Hilbert spaces provides a new Hilbert space.
\end{enumerate}

\subsubsection*{Spaces of structures and subspaces of the tensor algebra of a vector space}

There are also certain spaces of interesting structures on a vector
space that at least in the case of finite dimension correspond to
certain subspaces of the tensor algebra of the vector space.  These
spaces include:

\begin{enumerate}
\item The space of Euclidean inner products.
\item The space of Hermitian inner products.
\item the space of symplectic structures.
\item vector bundles
\item space of connections
\end{enumerate}
%%%%%
%%%%%
\end{document}
