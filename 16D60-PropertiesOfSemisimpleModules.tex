\documentclass[12pt]{article}
\usepackage{pmmeta}
\pmcanonicalname{PropertiesOfSemisimpleModules}
\pmcreated{2013-03-22 18:53:27}
\pmmodified{2013-03-22 18:53:27}
\pmowner{joking}{16130}
\pmmodifier{joking}{16130}
\pmtitle{properties of semisimple modules}
\pmrecord{4}{41738}
\pmprivacy{1}
\pmauthor{joking}{16130}
\pmtype{Theorem}
\pmcomment{trigger rebuild}
\pmclassification{msc}{16D60}

% this is the default PlanetMath preamble.  as your knowledge
% of TeX increases, you will probably want to edit this, but
% it should be fine as is for beginners.

% almost certainly you want these
\usepackage{amssymb}
\usepackage{amsmath}
\usepackage{amsfonts}

% used for TeXing text within eps files
%\usepackage{psfrag}
% need this for including graphics (\includegraphics)
%\usepackage{graphicx}
% for neatly defining theorems and propositions
%\usepackage{amsthm}
% making logically defined graphics
%%%\usepackage{xypic}

% there are many more packages, add them here as you need them

% define commands here

\begin{document}
Let $R$ be a ring. Recall that $R$-module $M$ is called \textit{semisimple} iff $M$ is a direct sum of simple module.

\textbf{Proposition.} The following are equivalent for $R$-module $M$:\begin{enumerate}\item $M$ is semisimple;
\item $M$ is generated by its simple submodules;
\item for every submodule $N\subseteq M$ there exists a submodule $N'\subseteq M$ such that $M=N\oplus N'$.
\end{enumerate}
%%%%%
%%%%%
\end{document}
