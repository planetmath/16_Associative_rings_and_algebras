\documentclass[12pt]{article}
\usepackage{pmmeta}
\pmcanonicalname{ComaximalIdeals}
\pmcreated{2013-03-22 12:35:57}
\pmmodified{2013-03-22 12:35:57}
\pmowner{yark}{2760}
\pmmodifier{yark}{2760}
\pmtitle{comaximal ideals}
\pmrecord{8}{32851}
\pmprivacy{1}
\pmauthor{yark}{2760}
\pmtype{Definition}
\pmcomment{trigger rebuild}
\pmclassification{msc}{16D25}
%\pmkeywords{comaximal}
\pmrelated{MaximalIdeal}
\pmdefines{comaximal}

\usepackage{amssymb}
\usepackage{amsmath}
\usepackage{amsfonts}

\begin{document}
\PMlinkescapeword{equivalent}

Let $R$ be a ring.

Two ideals $I$ and $J$ of $R$ are said to be \emph{comaximal} if $I + J = R$.
If $R$ is \PMlinkname{unital}{Ring}, this is equivalent to requiring that
there be $x\in I$ and $y\in J$ such that $x+y=1$.

For example, any two distinct maximal ideals of $R$ are comaximal.

A set $\cal S$ of ideals of $R$ is said to be \emph{pairwise comaximal} (or just \emph{comaximal}) if $I+J=R$ for all distinct $I,J\in\cal S$.
%%%%%
%%%%%
\end{document}
