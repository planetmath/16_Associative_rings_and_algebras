\documentclass[12pt]{article}
\usepackage{pmmeta}
\pmcanonicalname{GroupOfUnits}
\pmcreated{2013-03-22 14:41:32}
\pmmodified{2013-03-22 14:41:32}
\pmowner{pahio}{2872}
\pmmodifier{pahio}{2872}
\pmtitle{group of units}
\pmrecord{24}{36301}
\pmprivacy{1}
\pmauthor{pahio}{2872}
\pmtype{Theorem}
\pmcomment{trigger rebuild}
\pmclassification{msc}{16U60}
\pmclassification{msc}{13A05}
\pmsynonym{unit group}{GroupOfUnits}
\pmrelated{CommutativeRing}
\pmrelated{DivisibilityInRings}
\pmrelated{NonZeroDivisorsOfFiniteRing}
\pmrelated{PrimeResidueClass}
\pmdefines{group of units of ring}
\pmdefines{multiplicative group of field}

% this is the default PlanetMath preamble.  as your knowledge
% of TeX increases, you will probably want to edit this, but
% it should be fine as is for beginners.

% almost certainly you want these
\usepackage{amssymb}
\usepackage{amsmath}
\usepackage{amsfonts}

% used for TeXing text within eps files
%\usepackage{psfrag}
% need this for including graphics (\includegraphics)
%\usepackage{graphicx}
% for neatly defining theorems and propositions
 \usepackage{amsthm}
% making logically defined graphics
%%%\usepackage{xypic}

% there are many more packages, add them here as you need them

% define commands here
\theoremstyle{definition}
\newtheorem*{thmplain}{Theorem}

\begin{document}
\begin{thmplain}
 \, The set $E$ of units of a ring $R$ forms a group with respect to ring multiplication. 
\end{thmplain}

{\em Proof.}\, If $u$ and $v$ are two units, then there are the elements $r$ and $s$ of $R$ such that\, $ru = ur = 1$\, and\, $sv = vs = 1$.\, Then we get that\,$(sr)(uv) = s(r(uv)) = s((ru)v) = s(1v) = sv = 1$,\, similarly\, $(uv)(sr) = 1$.\, Thus also $uv$ is a unit, which means that $E$ is closed under multiplication.\, Because\, $1 \in E$\, and along with $u$ also its inverse $r$ belongs to $E$, the set $E$ is a group.\\

\textbf{Corollary.}\, In a commutative ring, a ring product is a unit iff all \PMlinkescapetext{factors} are units.\\

The group $E$ of the units of the ring $R$ is called the {\it group of units of the ring}.\, If $R$ is a field, $E$ is said to be the {\it multiplicative group of the field}.\\


\textbf{Examples}
\begin{enumerate}
 \item When\, $R = \mathbb{Z}$, then\, $E = \{1,\,-1\}$.
 \item When\, $R = \mathbb{Z}[i]$,\, the ring of Gaussian integers, then\, 
$E = \{1,\,i,\,-1,\,-i\}$.
 \item When\, $R = \mathbb{Z}[\sqrt{3}]$, \PMlinkname{then}{UnitsOfQuadraticFields}\, 
$E = \{\pm(2\!+\!\sqrt{3})^n\,\vdots\,\,\, n\in\mathbb{Z}\}$.
 \item When\, $R = K[X]$\, where $K$ is a field, then\, $E = K\!\smallsetminus\!\{0\}$.
 \item When\, 
$R = \{0\!+\!\mathbb{Z},\,1\!+\!\mathbb{Z},\,\ldots,\, 
 m\!-\!1\!+\!\mathbb{Z}\}$\, is the residue class ring modulo $m$, then\, $E$ consists of the prime classes modulo $m$, i.e. the residue classes $l\!+\!\mathbb{Z}$ satisfying\, $\gcd(l, m) = 1$.
\end{enumerate}

%%%%%
%%%%%
\end{document}
