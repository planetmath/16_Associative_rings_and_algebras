\documentclass[12pt]{article}
\usepackage{pmmeta}
\pmcanonicalname{GoldiesTheorem}
\pmcreated{2013-03-22 14:04:17}
\pmmodified{2013-03-22 14:04:17}
\pmowner{mclase}{549}
\pmmodifier{mclase}{549}
\pmtitle{Goldie's theorem}
\pmrecord{7}{35429}
\pmprivacy{1}
\pmauthor{mclase}{549}
\pmtype{Theorem}
\pmcomment{trigger rebuild}
\pmclassification{msc}{16U20}
\pmclassification{msc}{16P60}
\pmrelated{OresTheorem2}

% this is the default PlanetMath preamble.  as your knowledge
% of TeX increases, you will probably want to edit this, but
% it should be fine as is for beginners.

% almost certainly you want these
\usepackage{amssymb}
\usepackage{amsmath}
\usepackage{amsfonts}

% used for TeXing text within eps files
%\usepackage{psfrag}
% need this for including graphics (\includegraphics)
%\usepackage{graphicx}
% for neatly defining theorems and propositions
%\usepackage{amsthm}
% making logically defined graphics
%%%\usepackage{xypic}

% there are many more packages, add them here as you need them

% define commands here
\begin{document}
Let $R$ be a ring with an identity.  Then $R$ has a right classical ring of quotients $Q$ which is semisimple Artinian if and only if $R$ is a semiprime right Goldie ring.  If this is the case, then the composition length of $Q$ is equal to the uniform dimension of $R$.

An immediate corollary of this is that a semiprime right Noetherian ring always has a right classical ring of quotients.

This result was discovered by Alfred \PMlinkescapetext{Goldie} in the late 1950's.
%%%%%
%%%%%
\end{document}
