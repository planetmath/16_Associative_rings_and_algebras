\documentclass[12pt]{article}
\usepackage{pmmeta}
\pmcanonicalname{DenseRingOfLinearTransformations}
\pmcreated{2013-03-22 15:38:46}
\pmmodified{2013-03-22 15:38:46}
\pmowner{CWoo}{3771}
\pmmodifier{CWoo}{3771}
\pmtitle{dense ring of linear transformations}
\pmrecord{10}{37578}
\pmprivacy{1}
\pmauthor{CWoo}{3771}
\pmtype{Definition}
\pmcomment{trigger rebuild}
\pmclassification{msc}{16K40}
\pmsynonym{dense ring}{DenseRingOfLinearTransformations}
\pmrelated{SchursLemma}
\pmdefines{Jacobson Density Theorem}

\endmetadata

\usepackage{amssymb,amscd}
\usepackage{amsmath}
\usepackage{amsfonts}
\usepackage{amsthm}

% used for TeXing text within eps files
%\usepackage{psfrag}
% need this for including graphics (\includegraphics)
%\usepackage{graphicx}
% for neatly defining theorems and propositions
%\usepackage{amsthm}
% making logically defined graphics
%%%\usepackage{xypic}

% define commands here
\begin{document}
Let $D$ be a division ring and $V$ a vector space over $D$.  Let $R$ be a subring of the ring of endomorphisms (linear transformations) $\operatorname{End}_D(V)$ of $V$.  Then $R$ is said to be a \emph{dense ring of linear transformations} (over $D$) if we are given 
\begin{enumerate}
\item any positive integer $n$, 
\item any set $\lbrace v_1,\ldots,v_n\rbrace$ of linearly independent vectors in $V$, and 
\item any set $\lbrace w_i,\ldots,w_n\rbrace$ of arbitrary vectors in $V$,
\end{enumerate}
then there exists an element $f\in R$ such that
$$f(v_i)=w_i\quad\mbox{ for }i=1,\ldots,n.$$

Note that the linear independence of the $v_i$'s is essential in insuring the existence of a linear transformation $f$.  Otherwise, suppose $0=\sum d_iv_i$ where $d_1\neq 0$.  Pick $w_i$'s so that they are linearly independent.  Then $0=f(\sum d_iv_i)=\sum d_if(v_i)=\sum d_iw_i$, contradicting the linear independence of the $w_i$'s.

The notion of ``dense'' comes from topology: if $V$ is given the discrete topology and $\operatorname{End}_D(V)$ the compact-open topology, then $R$ is dense in $\operatorname{End}_D(V)$ iff $R$ is a dense ring of linear transformations of $V$.

\begin{proof} First, assume that $R$ is a dense ring of linear transformations of $V$.  Recall that the compact-open topology on $\operatorname{End}_D(V)$ has subbasis of the form $B(K,U):=\lbrace f\mid f(K)\subseteq U\rbrace$, where $U$ is open and $K$ is compact in $V$.  Since $V$ is discrete, $K$ is finite.  Now, pick a point $g\in \operatorname{End}_D(V)$ and let $$B=\bigcup_{\alpha\in I}\bigcap_{i=1}^{n(\alpha)}B(K_{i\alpha},U_{i\alpha})$$ be a neighborhood of $g$, $I$ some index set.  Then for some $\alpha\in I$, $g\in \bigcap B(K_{i\alpha},U_{i\alpha})$.  This means that $g(K_{i\alpha})\subseteq U_{i\alpha}$ for all $i=1,\ldots,n(\alpha)$.  Since each $K_{i\alpha}$ is finite, so is $K:=\bigcup K_{i\alpha}$.  After some re-indexing, let $\lbrace v_1,\ldots, v_n\rbrace$ be a maximal linearly independent subset of $K$.  Set $w_j=g(v_j)$, $j=1,\ldots,n$.  By assumption, there is an $f\in R$ such that $f(v_j)=w_j$, for all $j$.  For any $v\in K$, $v$ is a linear combination of the $v_j$'s: $v=\sum d_jv_j$, $d_j\in D$.  Then $f(v)=\sum d_jf(v_j)=\sum d_jg(_j)=g(v)\in U_{i\alpha}$ for some $i$.  This shows that $f(K_{i\alpha})\subseteq U_{i\alpha}$ and we have $f\in \bigcap B(K_{i\alpha},U_{i\alpha})\subseteq B$.

Conversely, assume that the ring $R$ is a dense subset of the space $\operatorname{End}_D(V)$.  Let $v_1,\ldots,v_n$ be linearly independent, and $w_1,\ldots,w_n$ be arbitrary vectors in $V$.  Let $W$ be the subspace spanned by the $v_i$'s.  Because the $v_i$'s are linearly independent, there exists a linear transformation $g$ such that $g(v_i)=w_i$ and $g(v)=0$ for $v\notin W$.  Let $K_i=\lbrace v_i\rbrace$ and $U_i=\lbrace w_i\rbrace$.  Then the $K_i$'s are compact and the $U_i$'s are open in the discrete space $V$.  Clearly $g\in\lbrace h\mid h(v_i)=w_i\rbrace=B(K_i,U_i)$ for each $i=1,\ldots,n$.  So $g$ lies in the neighborhood $B=\cap B(K_i,U_i)\subseteq \operatorname{End}_D(V)$.  Since $R$ is dense in $\operatorname{End}_D(V)$, there is an $f\in R\cap B$.  This implies that $f(v_i)=w_i$ for all $i$.
\end{proof}

\textbf{Remarks.}
\begin{itemize}
\item If $V$ is finite dimensional over $D$, then any dense ring of linear transformations $R=\operatorname{End}_D(V)$.  This can be easily observed by using the second half of the proof above.  Take a basis $v_1,\ldots,v_n$ of $V$ and any set of $n$ vectors $w_1,\ldots,w_n$ in $V$.  Let $g$ be the linear transformation that maps $v_i$ to $w_i$.  The above proof shows that there is an $f\in R$ such that $f$ agrees with $g$ on the basis elements.  But then they must agree on all of $V$ as a result, which is precisely the statement that $g=f\in R$.
\item It can be shown that a ring $R$ is a primitive ring iff it is isomorphic to a dense ring of linear transformations of a vector space over a division ring.  This is known as the \emph{\PMlinkescapetext{Jacobson Density Theorem}}.  It is a generalization of the special case of the Wedderburn-Artin Theorem when the ring in question is a simple Artinian ring.  In the general case, the finite chain condition is dropped.
\end{itemize}
%%%%%
%%%%%
\end{document}
