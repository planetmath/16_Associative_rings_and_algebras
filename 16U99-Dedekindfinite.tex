\documentclass[12pt]{article}
\usepackage{pmmeta}
\pmcanonicalname{Dedekindfinite}
\pmcreated{2013-03-22 14:18:23}
\pmmodified{2013-03-22 14:18:23}
\pmowner{CWoo}{3771}
\pmmodifier{CWoo}{3771}
\pmtitle{Dedekind-finite}
\pmrecord{10}{35766}
\pmprivacy{1}
\pmauthor{CWoo}{3771}
\pmtype{Definition}
\pmcomment{trigger rebuild}
\pmclassification{msc}{16U99}
\pmsynonym{von Neumann-finite}{Dedekindfinite}

\endmetadata

% this is the default PlanetMath preamble.  as your knowledge
% of TeX increases, you will probably want to edit this, but
% it should be fine as is for beginners.

% almost certainly you want these
\usepackage{amssymb}
\usepackage{amsmath}
\usepackage{amsfonts}

% used for TeXing text within eps files
%\usepackage{psfrag}
% need this for including graphics (\includegraphics)
%\usepackage{graphicx}
% for neatly defining theorems and propositions
%\usepackage{amsthm}
% making logically defined graphics
%%%\usepackage{xypic}

% there are many more packages, add them here as you need them

% define commands here
\begin{document}
A ring $R$ is \emph{Dedekind-finite} if for $a, b \in R$, whenever $ab=1$ implies $ba=1$.

Of course, every commutative ring is Dedekind-finite.  Therefore, the theory of Dedekind finiteness is trivial in this case.  Some other examples are 
\begin{enumerate}
\item any ring of endomorphisms over a finite dimensional vector space (over a field)
\item any division ring
\item any ring of matrices over a division ring
\item finite direct product of Dedekind-finite rings
\item by the last three examples, any semi-simple ring is Dedekind-finite.
\item any ring $R$ with the property that there is a natural number $n$ such that $x^n=0$ for every nilpotent element $x\in R$
\end{enumerate}

The finite dimensionality in the first example can not be extended to the infinite case.  Lam in \cite{lam} gave an example of a ring that is not Dedekind-finite arising out of the ring of endomorphisms over an infinite dimensional vector space (over a field).

\begin{thebibliography}{8}
\bibitem{lam} T. Y. Lam, {\em A First Course in Noncommutative Rings}, Springer-Verlag, New York (1991).
\bibitem{lam1} T. Y. Lam, {\em Lectures on Modules and Rings}, Springer-Verlag, New York (1999).
\end{thebibliography}
%%%%%
%%%%%
\end{document}
