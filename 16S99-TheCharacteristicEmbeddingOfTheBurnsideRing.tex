\documentclass[12pt]{article}
\usepackage{pmmeta}
\pmcanonicalname{TheCharacteristicEmbeddingOfTheBurnsideRing}
\pmcreated{2013-03-22 18:08:09}
\pmmodified{2013-03-22 18:08:09}
\pmowner{joking}{16130}
\pmmodifier{joking}{16130}
\pmtitle{the characteristic embedding of the Burnside ring}
\pmrecord{7}{40688}
\pmprivacy{1}
\pmauthor{joking}{16130}
\pmtype{Derivation}
\pmcomment{trigger rebuild}
\pmclassification{msc}{16S99}

% this is the default PlanetMath preamble.  as your knowledge
% of TeX increases, you will probably want to edit this, but
% it should be fine as is for beginners.

% almost certainly you want these
\usepackage{amssymb}
\usepackage{amsmath}
\usepackage{amsfonts}

% used for TeXing text within eps files
%\usepackage{psfrag}
% need this for including graphics (\includegraphics)
%\usepackage{graphicx}
% for neatly defining theorems and propositions
%\usepackage{amsthm}
% making logically defined graphics
%%%\usepackage{xypic}

% there are many more packages, add them here as you need them

% define commands here

\begin{document}
Let $G$ be a finite group, $H$ its subgroup and $X$ a finite $G$-set. By the $H$\textit{-fixed point subset} of $X$ we understand the set $$X^{H}=\{x\in X;\ \forall_{h\in H}\ hx=x\}.$$
Denote by $\vert X\vert$ the cardinality of a set $X$.\\ \\
It is easy to see that for any $G$-sets $X,Y$ we have:
$$\vert (X\sqcup Y)^{H}\vert =\vert X^{H}\vert +\vert Y^{H}\vert ;$$
$$\vert (X\times Y)^{H}\vert =\vert X^{H}\vert\cdot\vert Y^{H}\vert.$$
Denote by $\mathrm{Sub}(G)=\{H\subseteq G;\ H\ \mathrm{is}\ \mathrm{a}\ \mathrm{subgroup}\ \mathrm{of}\ G\}$. Recall that any $H,K\in\mathrm{Sub}(G)$ are said to be conjugate iff there exists $g\in G$ such that $H=gKg^{-1}$. Conjugation is an equivalence relation. Denote by $\mathrm{Con}(G)$ the quotient set.\\ \\
One can check that for any $H,K\in\mathrm{Sub}(G)$ such that $H$ is conjugate to $K$ and for any finite $G$-set $X$ we have
$$\vert X^{H}\vert = \vert X^{K}\vert.$$
Thus we have a well defined ring homomorphism:
$$\varphi:\Omega(G)\rightarrow\bigoplus_{(H)\in\mathrm{Con}(G)}\mathbb{Z};$$
$$\varphi([X]-[Y])=(\vert X^{H}\vert - \vert Y^{H}\vert)_{(H)\in\mathrm{Con}(G)}.$$
This homomorphism is known as the characteristic embedding, since it is monomorphism (see \cite{T} for proof).\\[20pt]
\begin{thebibliography}{1}
\bibitem{T} T. tom Dieck, {\it Transformation groups and representation theory}, Lecture Notes in Math. 766, Springer-Verlag, Berlin, 1979.
\end{thebibliography}


%%%%%
%%%%%
\end{document}
