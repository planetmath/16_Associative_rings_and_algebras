\documentclass[12pt]{article}
\usepackage{pmmeta}
\pmcanonicalname{GrouplikeElementsInHopfAlgebras}
\pmcreated{2013-03-22 18:58:39}
\pmmodified{2013-03-22 18:58:39}
\pmowner{joking}{16130}
\pmmodifier{joking}{16130}
\pmtitle{grouplike elements in Hopf algebras}
\pmrecord{5}{41841}
\pmprivacy{1}
\pmauthor{joking}{16130}
\pmtype{Definition}
\pmcomment{trigger rebuild}
\pmclassification{msc}{16W30}

% this is the default PlanetMath preamble.  as your knowledge
% of TeX increases, you will probably want to edit this, but
% it should be fine as is for beginners.

% almost certainly you want these
\usepackage{amssymb}
\usepackage{amsmath}
\usepackage{amsfonts}

% used for TeXing text within eps files
%\usepackage{psfrag}
% need this for including graphics (\includegraphics)
%\usepackage{graphicx}
% for neatly defining theorems and propositions
%\usepackage{amsthm}
% making logically defined graphics
%%%\usepackage{xypic}

% there are many more packages, add them here as you need them

% define commands here

\begin{document}
Recall, that if $k$ is a field and $G$ is a group, then the group algebra $kG$ can be turned into a Hopf algebra, by defining comultiplication $\Delta(g)=g\otimes g$, counit $\varepsilon(g)=1$ and antipode $S(g)=g^{-1}$.

Now let $H$ be a Hopf algebra over a field $k$, with identity $1$, comultiplication $\Delta$, counit $\varepsilon$ and antipode $S$. Recall that element $g\in H$ is called grouplike iff $g\neq 0$ and $\Delta(g)=g\otimes g$. The set of all grouplike elements $G(H)$ is nonempty, because $1\in G(H)$. Also, since comultiplication is an algebra morphism, then $G(H)$ is multiplicative, i.e. if $g,h\in G(H)$, then $gh\in G(H)$. Furthermore, it can be shown that for any $g\in G(H)$ we have $S(g)\in G(H)$ and $S(g)g=gS(g)=1$. Thus $G(H)$ is a group under multiplication inherited from $H$.

It is easy to see, that the vector subspace spanned by $G(H)$ is a Hopf subalgebra of $H$ isomorphic to $kG(H)$. It can be shown that $G(H)$ is always linearly independent, so if $H$ is finite dimensional, then $G(H)$ is a finite group. Also, if $H$ is finite dimensional, then it follows from the Nichols-Zoeller Theorem, that the order of $G(H)$ divides $\mathrm{dim}_{k}H$. 

From these observations it follows that if $\mathrm{dim}_{k}H=p$ is a prime number, then $G(H)$ is either trivial or the order of $G(H)$ is equal to $p$ (i.e. $G(H)$ is cyclic of order $p$). The second case implies that $H$ is isomorphic to $k\mathbb{Z}_{p}$ and it can be shown that the first case cannot occur.
%%%%%
%%%%%
\end{document}
