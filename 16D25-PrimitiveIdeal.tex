\documentclass[12pt]{article}
\usepackage{pmmeta}
\pmcanonicalname{PrimitiveIdeal}
\pmcreated{2013-03-22 12:01:45}
\pmmodified{2013-03-22 12:01:45}
\pmowner{antizeus}{11}
\pmmodifier{antizeus}{11}
\pmtitle{primitive ideal}
\pmrecord{6}{31007}
\pmprivacy{1}
\pmauthor{antizeus}{11}
\pmtype{Definition}
\pmcomment{trigger rebuild}
\pmclassification{msc}{16D25}
\pmsynonym{primitive ring}{PrimitiveIdeal}

\endmetadata

\usepackage{amssymb}
\usepackage{amsmath}
\usepackage{amsfonts}
\usepackage{graphicx}
%%%\usepackage{xypic}
\begin{document}
Let $R$ be a ring, and let $I$ be an ideal of $R$.
We say that $I$ is a {\it left (right) primitive ideal}
if there exists a simple left (right) $R$-module $X$
such that $I$ is the annihilator of $X$ in $R$.

We say that $R$ is a {\it left (right) primitive ring}
if the zero ideal is a left (right) primitive ideal of $R$.

Note that $I$ is a left (right) primitive ideal 
if and only if $R/I$ is a left (right) primitive ring.
%%%%%
%%%%%
%%%%%
\end{document}
