\documentclass[12pt]{article}
\usepackage{pmmeta}
\pmcanonicalname{FreeModule1}
\pmcreated{2013-03-22 14:03:50}
\pmmodified{2013-03-22 14:03:50}
\pmowner{Mathprof}{13753}
\pmmodifier{Mathprof}{13753}
\pmtitle{free module}
\pmrecord{5}{35420}
\pmprivacy{1}
\pmauthor{Mathprof}{13753}
\pmtype{Definition}
\pmcomment{trigger rebuild}
\pmclassification{msc}{16D40}

% this is the default PlanetMath preamble.  as your knowledge
% of TeX increases, you will probably want to edit this, but
% it should be fine as is for beginners.

% almost certainly you want these
\usepackage{amssymb}
\usepackage{amsmath}
\usepackage{amsfonts}

% used for TeXing text within eps files
%\usepackage{psfrag}
% need this for including graphics (\includegraphics)
%\usepackage{graphicx}
% for neatly defining theorems and propositions
%\usepackage{amsthm}
% making logically defined graphics
%%%\usepackage{xypic} 

% there are many more packages, add them here as you need them

% define commands here
\begin{document}
Let $R$ be a ring.
A {\it free module} over $R$
is a direct sum of copies of $R$.

Similarly, as an abelian group
is simply a module over $\Bbb{Z}$,
a {\it free abelian group}
is a direct sum of copies of $\Bbb{Z}$.

This is equivalent to saying
that the module has a {\it free basis},
i.e. a set of elements
with the property
that every element of the module
can be uniquely expressed
as an linear combination over $R$
of elements of the free basis.

Every free module is also a projective module,
as well as a flat module.
%%%%%
%%%%%
\end{document}
