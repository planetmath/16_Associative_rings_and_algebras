\documentclass[12pt]{article}
\usepackage{pmmeta}
\pmcanonicalname{ExampleOfFreeModuleWithBasesOfDiffrentCardinality}
\pmcreated{2013-03-22 18:07:18}
\pmmodified{2013-03-22 18:07:18}
\pmowner{joking}{16130}
\pmmodifier{joking}{16130}
\pmtitle{example of free module with bases of diffrent cardinality}
\pmrecord{13}{40670}
\pmprivacy{1}
\pmauthor{joking}{16130}
\pmtype{Example}
\pmcomment{trigger rebuild}
\pmclassification{msc}{16D40}
\pmrelated{IBN}

\endmetadata

% this is the default PlanetMath preamble.  as your knowledge
% of TeX increases, you will probably want to edit this, but
% it should be fine as is for beginners.

% almost certainly you want these
\usepackage{amssymb}
\usepackage{amsmath}
\usepackage{amsfonts}

% used for TeXing text within eps files
%\usepackage{psfrag}
% need this for including graphics (\includegraphics)
%\usepackage{graphicx}
% for neatly defining theorems and propositions
%\usepackage{amsthm}
% making logically defined graphics
%%%\usepackage{xypic}

% there are many more packages, add them here as you need them

% define commands here

\begin{document}
Let $k$ be a field and $V$ be an infinite dimensional vector space over $k$. Let $\{e_i\}_{i\in I}$ be its basis. Denote by $R=\mathrm{End}(V)$ the ring of endomorphisms of $V$ with standard addition and composition as a multiplication.\\ \\
Let $J$ be any set such that $|J|\leq |I|$. \\ \\ 
\textbf{Proposition.} $R$ and $\prod_{j\in J}R$ are isomorphic as a $R$-modules.\\ \\ 
\textit{Proof.} Let $\alpha:I\rightarrow J\times I$ be a bijection (it exists since $|I|\geq |J|$ and $I$ is infinite) and denote by $\pi_1:J\times I\rightarrow J$ and $\pi_2:J\times I\rightarrow I$ the projections. Moreover let $\delta_1=\pi_1 \circ \alpha$ and $\delta_2=\pi_2 \circ \alpha$. 

Recall that $\prod_{j\in J}R = \{f:J\to R\}$ (with obvious $R$-module structure) and define a map $\phi:\prod_{j\in J}R\rightarrow R$ by defining the endomorphism $\phi(f)\in R$ for $f\in\prod_{j\in J}R$ as follows:
$$\phi(f)(e_i)=f(\delta_1 (i))(e_{\delta_2 (i)}).$$ $ $ \\ 
We will show that $\phi$ is an isomorphism. It is easy to see that $\phi$ is a $R$-module homomorphism. Therefore it is enough to show that $\phi$ is injective and surjective.

$1)$ Recall that $\phi$ is injective if and only if $\mathrm{ker}(\phi)=0$. So assume that $\phi(f)=0$ for $f\in\prod_{j\in J}R$. Note that $f=0$ if and only if $f(j)=0$ for all $j\in J$ and this is if and only if $f(j)(e_{i})=0$ for all $j\in J$ and $i\in I$. So take any $(j,i)\in J\times I$. Then (since $\alpha$ is bijective) there exists $i_0\in I$ such that $\alpha(i_0)=(j,i)$. It follows that $\delta_1 (i_0)=j$ and $\delta_2 (i_0)=i$. Thus we have
$$0=\phi(f)(e_{i_{0}})=f(\delta_1 (i_0))(e_{\delta_2 (i_0)})=f(j)(e_i).$$
Since $j$ and $i$ were arbitrary, then $f=0$ which completes this part.

$2)$ We wish to show that $\phi$ is onto, so take any $h\in R$. Define $f\in\prod_{j\in J} R$ by the following formula:
$$f(j)(e_i)=h(e_{\alpha^{-1}(j,i)}).$$
It is easy to see that $\phi(f)=h$. $\square$ \\ \\
\textbf{Corollary.} For any two numbers $n,m\in\mathbb{N}$ there exists a ring $R$ and a free module $M$ such that $M$ has two bases with cardinality $n,m$ respectively.\\ \\
\textit{Proof.} It follows from the proposition, that for $R=\mathrm{End}(V)$ we have $$R^{n}\simeq R\simeq R^{m}.$$ For finite set $J$ module $\prod_{j\in J} R$ is free with basis consisting $|J|$ elements (product is the same as direct sum). Therefore (due to existence of previous isomorphisms) $R$-module $R$ has two bases, one of cardinality $n$ and second of cardinality $m$. $\square$
%%%%%
%%%%%
\end{document}
