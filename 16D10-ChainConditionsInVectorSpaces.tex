\documentclass[12pt]{article}
\usepackage{pmmeta}
\pmcanonicalname{ChainConditionsInVectorSpaces}
\pmcreated{2013-03-22 19:11:55}
\pmmodified{2013-03-22 19:11:55}
\pmowner{rm50}{10146}
\pmmodifier{rm50}{10146}
\pmtitle{chain conditions in vector spaces}
\pmrecord{4}{42112}
\pmprivacy{1}
\pmauthor{rm50}{10146}
\pmtype{Theorem}
\pmcomment{trigger rebuild}
\pmclassification{msc}{16D10}
%\pmkeywords{Noetherian}
%\pmkeywords{Artinian}

\usepackage{amssymb}
\usepackage{amsmath}
\usepackage{amsfonts}

% used for TeXing text within eps files
%\usepackage{psfrag}
% need this for including graphics (\includegraphics)
%\usepackage{graphicx}
% for neatly defining theorems and propositions
\usepackage{amsthm}
% making logically defined graphics
%%%\usepackage{xypic}

% there are many more packages, add them here as you need them

% define commands here
\newcommand{\BQ}{\mathbb{Q}}
\newcommand{\BR}{\mathbb{R}}
\newcommand{\BZ}{\mathbb{Z}}
\newtheorem{thm}{Theorem}
\newtheorem{cor}{Corollary}
\newcommand{\smm}{\mathfrak{m}}
\begin{document}
\PMlinkescapeword{parent}
From the theorem in the parent article - that an $A$-module $M$ has a composition series if and only if it satisfies both chain conditions - it is easy to see that
\begin{thm} Let $k$ be a field, $V$ a $k$-vector space. Then the following are equivalent:
\begin{enumerate}
\item $V$ is finite-dimensional;
\item $V$ has a composition series;
\item $V$ satisfies the ascending chain condition (acc);
\item $V$ satisfies the descending chain condition (dcc).
\end{enumerate}
\end{thm}
\begin{proof}
Clearly (1) $\Rightarrow$ (2), since submodules are just subspaces. (2) $\Rightarrow$ (3) and (2) $\Rightarrow$ (4) from the parent article. So it remains to see that (3) $\Rightarrow$ (1) and (4) $\Rightarrow$ (1). But if $V$ is infinite-dimensional, we can choose a sequence $\{x_i\}_{i\ge 1}$ of linearly independent elements. Let $U_n$ be the subspace spanned by $x_1,\dotsc,x_n$ and $V_n$ the subspace spanned by $x_{n+1},x_{n+2},\dots$. Then the $U_i$ form a strictly ascending infinite family of subspaces, so $V$ does not satisfy the ascending chain condition; the $V_i$ form a strictly descending infinite family of subspaces, so $V$ does not satisfy the descending chain condition.
\end{proof}

This easily implies the following:
\begin{cor} Let $A$ be a ring in which $(0) = \smm_1\dots\smm_n$ where the $\smm_i$ are (not necessarily distinct) maximal ideals. Then $A$ is Noetherian if and only if $A$ is Artinian.
\end{cor}
\begin{proof} We have the sequence of ideals
\[
  A \supset \smm_1 \supset \smm_1\smm_2 \supset\dots\supset \smm_1\dots\smm_n = 0
\]
Each factor $\smm_1\dots\smm_{i-1}/\smm_1\dots\smm_i$ is a vector space over the field $A/\smm_i$. By the above theorem, each quotient satisfies the acc if and only if it satisfies the dcc. But by repeatedly applying the fact that in a short exact sequence, the middle term satisfies the acc (dcc) if and only if both ends do, we see that $A$ satisfies the acc if and only if it satisfies the dcc.
\end{proof}
\begin{thebibliography}{10}
\bibitem{bib:am}
M.F.~Atiyah, I.G.~MacDonald, \emph{Introduction to Commutative Algebra}, Addison-Wesley 1969.
\end{thebibliography}

%%%%%
%%%%%
\end{document}
