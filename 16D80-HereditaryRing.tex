\documentclass[12pt]{article}
\usepackage{pmmeta}
\pmcanonicalname{HereditaryRing}
\pmcreated{2013-03-22 14:48:50}
\pmmodified{2013-03-22 14:48:50}
\pmowner{CWoo}{3771}
\pmmodifier{CWoo}{3771}
\pmtitle{hereditary ring}
\pmrecord{9}{36473}
\pmprivacy{1}
\pmauthor{CWoo}{3771}
\pmtype{Definition}
\pmcomment{trigger rebuild}
\pmclassification{msc}{16D80}
\pmclassification{msc}{16E60}
\pmdefines{hereditary module}

\endmetadata

% this is the default PlanetMath preamble.  as your knowledge
% of TeX increases, you will probably want to edit this, but
% it should be fine as is for beginners.

% almost certainly you want these
\usepackage{amssymb,amscd}
\usepackage{amsmath}
\usepackage{amsfonts}

% used for TeXing text within eps files
%\usepackage{psfrag}
% need this for including graphics (\includegraphics)
%\usepackage{graphicx}
% for neatly defining theorems and propositions
%\usepackage{amsthm}
% making logically defined graphics
%%%\usepackage{xypic}

% there are many more packages, add them here as you need them

% define commands here
\begin{document}
\PMlinkescapeword{hereditary}

Let $R$ be a ring.  A right (left) $R$-module $M$ is called right (left) \emph{hereditary} if every submodule of $M$ is projective over $R$.

\textbf{Remarks}.  
\begin{itemize}
\item If $M$ is semisimple, then $M$ is hereditary.
\item Suppose $M$ is an external direct sum of hereditary right (left) $R$-modules, then $M$ is itself hereditary.
\end{itemize}

A ring $R$ is said to be a right (left) \emph{hereditary ring} if all of its right (left) ideals are projective as modules over $R$.  If $R$ is both left and right hereditary, then $R$ is simply called a hereditary ring.

\textbf{Remarks.}
\begin{itemize}
\item Even though the notions of left and right heredity in rings are symmetrical, one does not imply the other.
\item If $R$ is semisimple, then $R$ is hereditary.
\item If $R$ is hereditary, then every free $R$-module is a hereditary module.
\item A hereditary integral domain is a Dedekind domain, and conversely.
\item The global dimension of a non-semisimple hereditary ring is 1.
\end{itemize}
%%%%%
%%%%%
\end{document}
