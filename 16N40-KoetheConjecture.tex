\documentclass[12pt]{article}
\usepackage{pmmeta}
\pmcanonicalname{KoetheConjecture}
\pmcreated{2013-03-22 13:13:27}
\pmmodified{2013-03-22 13:13:27}
\pmowner{mclase}{549}
\pmmodifier{mclase}{549}
\pmtitle{Koethe conjecture}
\pmrecord{5}{33691}
\pmprivacy{1}
\pmauthor{mclase}{549}
\pmtype{Conjecture}
\pmcomment{trigger rebuild}
\pmclassification{msc}{16N40}
\pmrelated{NilAndNilpotentIdeals}
\pmrelated{PropertiesOfNilAndNilpotentIdeals}

% this is the default PlanetMath preamble.  as your knowledge
% of TeX increases, you will probably want to edit this, but
% it should be fine as is for beginners.

% almost certainly you want these
\usepackage{amssymb}
\usepackage{amsmath}
\usepackage{amsfonts}

% used for TeXing text within eps files
%\usepackage{psfrag}
% need this for including graphics (\includegraphics)
%\usepackage{graphicx}
% for neatly defining theorems and propositions
%\usepackage{amsthm}
% making logically defined graphics
%%%\usepackage{xypic}

% there are many more packages, add them here as you need them

% define commands here
\begin{document}
The Koethe Conjecture is the statement that for any pair of nil right ideals $A$ and $B$ in any ring $R$, the sum $A + B$ is also nil.

If either of $A$ or $B$ is a two-sided ideal, it is easy to see that $A + B$ is nil.  (See properties of nil and nilpotent ideals.)

In particular, this means that the  Koethe conjecture is true for commutative rings.

It has been shown to be true for many classes of rings, but the general statement is still unproven, and no counter example has been found.
%%%%%
%%%%%
\end{document}
