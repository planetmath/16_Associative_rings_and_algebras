\documentclass[12pt]{article}
\usepackage{pmmeta}
\pmcanonicalname{ConstructionOfAnInjectiveResolution}
\pmcreated{2013-03-22 17:11:02}
\pmmodified{2013-03-22 17:11:02}
\pmowner{guffin}{12505}
\pmmodifier{guffin}{12505}
\pmtitle{construction of an injective resolution}
\pmrecord{5}{39500}
\pmprivacy{1}
\pmauthor{guffin}{12505}
\pmtype{Derivation}
\pmcomment{trigger rebuild}
\pmclassification{msc}{16E05}

\endmetadata

% this is the default PlanetMath preamble.  as your knowledge
% of TeX increases, you will probably want to edit this, but
% it should be fine as is for beginners.

% almost certainly you want these
\usepackage{amssymb}
\usepackage{amsmath}
\usepackage{amsfonts}

% used for TeXing text within eps files
%\usepackage{psfrag}
% need this for including graphics (\includegraphics)
%\usepackage{graphicx}
% for neatly defining theorems and propositions
%\usepackage{amsthm}
% making logically defined graphics
%%%\usepackage{xypic}

% there are many more packages, add them here as you need them

% define commands here

\begin{document}
The category of modules has enough injectives.
Let $M$ be a module, and let $I^0$ be an injective module such that 

\[
0  \longrightarrow M \longrightarrow I^0
\]

is exact.  Then, let $M_0$ be the image of $M$ in $I^0$, and construct the factor module $I^0\slash M^0$.  Then, since the category of modules has enough injectives, we can find a module $I^1$ such that 

\[
0 \longrightarrow I^0 \slash M^0 \stackrel{\phi_0}{\longrightarrow} I^1
\]

is exact.  $\phi_0$ induces a homomorphism $\phi \!:\! I^0 \longrightarrow I^1$, whose kernel is $M^0$.  We thus have an exact sequence

\[
0 \longrightarrow M \longrightarrow I^0 \longrightarrow I^1.
\]

One can continue this process to construct injective modules $I^n$ for any $n\in \mathbb Z$ (the resolution may terminate: $I^m = 0$ for some $N\in \mathbb Z$ with all $m > N$). 
%%%%%
%%%%%
\end{document}
