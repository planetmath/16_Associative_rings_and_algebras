\documentclass[12pt]{article}
\usepackage{pmmeta}
\pmcanonicalname{RadicalTheory}
\pmcreated{2013-03-22 13:13:02}
\pmmodified{2013-03-22 13:13:02}
\pmowner{mclase}{549}
\pmmodifier{mclase}{549}
\pmtitle{radical theory}
\pmrecord{10}{33682}
\pmprivacy{1}
\pmauthor{mclase}{549}
\pmtype{Definition}
\pmcomment{trigger rebuild}
\pmclassification{msc}{16N80}
\pmrelated{JacobsonRadical}
\pmdefines{radical}
\pmdefines{radical property}
\pmdefines{semisimple}
\pmdefines{hereditary}
\pmdefines{hereditary radical}
\pmdefines{supernilpotent}
\pmdefines{supernilpotent radical}

% this is the default PlanetMath preamble.  as your knowledge
% of TeX increases, you will probably want to edit this, but
% it should be fine as is for beginners.

% almost certainly you want these
\usepackage{amssymb}
\usepackage{amsmath}
\usepackage{amsfonts}

% used for TeXing text within eps files
%\usepackage{psfrag}
% need this for including graphics (\includegraphics)
%\usepackage{graphicx}
% for neatly defining theorems and propositions
%\usepackage{amsthm}
% making logically defined graphics
%%%\usepackage{xypic}

% there are many more packages, add them here as you need them

% define commands here

\newcommand{\xrad}{\mathcal{X}}
\begin{document}
\PMlinkescapeword{satisfies}
\PMlinkescapeword{independent}
\PMlinkescapeword{closed}
\PMlinkescapeword{theory}
\PMlinkescapeword{extensions}
\PMlinkescapeword{images}

Let $\xrad$ represent a property which a ring may or may not have.  This property may be anything at all: what is important is that for any ring $R$, the statement ``$R$ has property $\xrad$'' is either true or false.

We say that a ring which has the property $\xrad$ is an $\xrad$-ring.  An ideal $I$ of a ring $R$ is called an $\xrad$-ideal if, as a ring, it is an $\xrad$-ring.  (Note that this definition only makes sense if rings are not required to have identity elements; otherwise and ideal is not, in general, a ring.  Rings are not required to have an identity element in radical theory.)

The property $\xrad$ is a \emph{radical property} if it satisfies:
\begin{enumerate}
\item The class of $\xrad$-rings is closed under homomorphic images.
\item Every ring $R$ has a largest $\xrad$-ideal, which contains all other $\xrad$-ideals of $R$.  This ideal is written $\xrad(R)$.
\item $\xrad(R/\xrad(R)) = 0$.
\end{enumerate}

The ideal $\xrad(R)$ is called the \emph{$\xrad$-radical} of $R$.  A ring is called \emph{$\xrad$-radical} if $\xrad(R) = R$, and is called \emph{$\xrad$-semisimple} if $\xrad(R) = 0$.

If $\xrad$ is a radical property, then the class of $\xrad$-rings is also called the class of \emph{$\xrad$-radical rings}.

The class of $\xrad$-radical rings is closed under ideal extensions.  That is, if $A$ is an ideal of $R$, and $A$ and $R/A$ are $\xrad$-radical, then so is $R$.

\emph{Radical theory} is the study of radical properties and their interrelations.  There are several well-known radicals which are of independent interest in ring theory (See examples -- to follow).

The class of all radicals is however very large.  Indeed, it is possible to show that any partition of the class of simple rings into two classes $\mathcal{R}$ and $\mathcal{S}$ such that isomorphic simple rings are in the same class,  gives rise to a radical $\xrad$ with the property that all rings in $\mathcal{R}$ are $\xrad$-radical and all rings in $\mathcal{S}$ are $\xrad$-semisimple.  In fact, there are at least two distinct radicals for each such partition.

A radical $\xrad$ is \emph{hereditary} if every ideal of an $\xrad$-radical ring is also $\xrad$-radical.

A radical $\xrad$ is \emph{supernilpotent} if the class of $\xrad$-rings contains all nilpotent rings.

\section{Examples}

Nil is a radical property.  This property defines the nil radical, $\mathcal{N}$.

Nilpotency is not a radical property.

Quasi-regularity is a radical property.  The associated radical is the Jacobson radical, $\mathcal{J}$.
%%%%%
%%%%%
\end{document}
