\documentclass[12pt]{article}
\usepackage{pmmeta}
\pmcanonicalname{FinitelyGeneratedModule}
\pmcreated{2013-03-22 14:01:08}
\pmmodified{2013-03-22 14:01:08}
\pmowner{Thomas Heye}{1234}
\pmmodifier{Thomas Heye}{1234}
\pmtitle{finitely generated module}
\pmrecord{15}{34957}
\pmprivacy{1}
\pmauthor{Thomas Heye}{1234}
\pmtype{Definition}
\pmcomment{trigger rebuild}
\pmclassification{msc}{16D10}
%\pmkeywords{finitely generated module}
%\pmkeywords{span}
%\pmkeywords{cyclic module}
%\pmkeywords{zero vector $\vec{0}$}
%\pmkeywords{singleton}
\pmrelated{ModuleFinite}
\pmrelated{Span}
\pmdefines{finitely generated}
\pmdefines{cyclic module}

\usepackage{amssymb}
\usepackage{amsmath}
\usepackage{amsfonts}
\begin{document}
\PMlinkescapeword{cyclic}
A module $X$ over a ring $R$ is said to be {\em finitely generated} if there is a finite subset $Y$ of $X$ such that $Y$ spans $X$. Let us recall that the span of a (not necessarily finite) set $X$  of vectors is the class of all (finite) linear combinations of elements of $S$; moreover, let us recall that the span of the empty set is defined to be the singleton consisting of only one vector, the zero vector $\vec{0}$. A module $X$ is then called \emph{cyclic} if it can be \PMlinkescapetext{spanned by} a singleton. 
\par
\textbf{Examples}.  Let $R$ be a commutative ring with 1 and $x$ be an indeterminate.
\begin{enumerate}
\item $Rx=\lbrace rx \mid r\in R \rbrace$ is a cyclic $R$-module generated by $\lbrace x \rbrace$.
\item $R\oplus Rx$ is a finitely-generated $R$-module generated by $\lbrace 1, x \rbrace$.  Any element in $R\oplus Rx$ 
can be expressed uniquely as $r+sx$.
\item $R[x]$ is not finitely generated as an $R$-module.  For if there is a finite set $Y$ \PMlinkescapetext{spanning} $R[x]$, taking $d$ to be the largest of all degrees of polynomials in $Y$, then $x^{d+1}$ would not be in the \PMlinkescapetext{spanning set} of $Y$, assumed to be $R[x]$, which is a contradiction.  (Note, however, that $R[x]$ is finitely-generated as an $R$-algebra.)
\end{enumerate}
%%%%%
%%%%%
\end{document}
