\documentclass[12pt]{article}
\usepackage{pmmeta}
\pmcanonicalname{ReducedRing}
\pmcreated{2013-03-22 14:18:12}
\pmmodified{2013-03-22 14:18:12}
\pmowner{CWoo}{3771}
\pmmodifier{CWoo}{3771}
\pmtitle{reduced ring}
\pmrecord{17}{35762}
\pmprivacy{1}
\pmauthor{CWoo}{3771}
\pmtype{Definition}
\pmcomment{trigger rebuild}
\pmclassification{msc}{16N60}
\pmsynonym{nilpotent-free}{ReducedRing}
%\pmkeywords{reduced}

% this is the default PlanetMath preamble.  as your knowledge
% of TeX increases, you will probably want to edit this, but
% it should be fine as is for beginners.

% almost certainly you want these
\usepackage{amssymb}
\usepackage{amsmath}
\usepackage{amsfonts}

% used for TeXing text within eps files
%\usepackage{psfrag}
% need this for including graphics (\includegraphics)
%\usepackage{graphicx}
% for neatly defining theorems and propositions
%\usepackage{amsthm}
% making logically defined graphics
%%%\usepackage{xypic}

% there are many more packages, add them here as you need them

% define commands here
\begin{document}
\PMlinkescapeword{reduced}

A ring $R$ is said to be a {\it reduced ring} if $R$ contains no non-zero nilpotent elements.  In other words, $r^2=0$ implies $r=0$ for any $r\in R$.

Below are some examples of reduced rings.

\begin{itemize}
\item A reduced ring is semiprime.

\item A ring is a \PMlinkname{domain}{CancellationRing} iff it is \PMlinkname{prime}{PrimeRing} and reduced.

\item A commutative semiprime ring is reduced.  In particular, all integral domains and Boolean rings are reduced.

\item Assume that $R$ is commutative, and let $A$ be the set of all nilpotent elements.  Then $A$ is an ideal of $R$, and that $R/A$ is reduced (for if $(r+A)^2=0$, then $r^2\in A$, so $r^2$, and consequently $r$, is nilpotent, or $r\in A$).

\end{itemize}

An example of a reduced ring with zero-divisors is $\mathbb{Z}^n$, with multiplication defined componentwise: $(a_1,\ldots,a_n)(b_1,\ldots,b_n):=(a_1b_1,\ldots, a_nb_n)$.  A ring of functions taking values in a reduced ring is also reduced.

Some prototypical examples of rings that are not reduced are $\mathbb{Z}_8$, since $4^2=0$, as well as any matrix ring over any ring; as illustrated by the instance below

$$
\begin{pmatrix}
0 & 1 \\
0 & 0 
\end{pmatrix}
\begin{pmatrix}
0 & 1 \\
0 & 0
\end{pmatrix}
=
\begin{pmatrix}
0 & 0 \\
0 & 0
\end{pmatrix}.
$$
%%%%%
%%%%%
\end{document}
