\documentclass[12pt]{article}
\usepackage{pmmeta}
\pmcanonicalname{JacobsonRadical}
\pmcreated{2013-03-22 12:36:11}
\pmmodified{2013-03-22 12:36:11}
\pmowner{yark}{2760}
\pmmodifier{yark}{2760}
\pmtitle{Jacobson radical}
\pmrecord{19}{32856}
\pmprivacy{1}
\pmauthor{yark}{2760}
\pmtype{Definition}
\pmcomment{trigger rebuild}
\pmclassification{msc}{16N20}
\pmrelated{Annihilator}
\pmrelated{RadicalOfAnIdeal}
\pmrelated{SimpleModule}
\pmrelated{Nilradical}
\pmrelated{RadicalTheory}
\pmrelated{QuasiRegularity}

% this is the default PlanetMath preamble.  as your knowledge
% of TeX increases, you will probably want to edit this, but
% it should be fine as is for beginners.

% almost certainly you want these
\usepackage{amssymb}
\usepackage{amsmath}
\usepackage{amsfonts}

% used for TeXing text within eps files
%\usepackage{psfrag}
% need this for including graphics (\includegraphics)
%\usepackage{graphicx}
% for neatly defining theorems and propositions
%\usepackage{amsthm}
% making logically defined graphics
%%%\usepackage{xypic}

% there are many more packages, add them here as you need them

% define commands here
\begin{document}
The {\em Jacobson radical} $J(R)$ of a unital ring $R$ is the intersection
of the annihilators of \PMlinkname{simple}{SimpleModule} left $R$-modules.

The following are alternative characterizations of the Jacobson radical $J(R)$:
\begin{enumerate}
\item The intersection of all left primitive ideals.
\item The intersection of all maximal left ideals.
\item The set of all $t \in R$ such that for all $r \in R$, $1-rt$ is
      left invertible (i.e. there exists $u$ such that $u(1-rt)=1$).
\item The largest ideal $I$ such that for all $v \in I$, $1-v$ is a
      unit in $R$.
\item (1) - (3) with ``left'' replaced by ``right'' and $rt$ replaced by $tr$.
\end{enumerate}

If $R$ is commutative and finitely generated, then 
\[
  J(R)=\{x \in R \mid x^n=0 \hbox{ for some } n \in \mathbb{N} \}
   = \operatorname{Nil}(R).
\]

The Jacobson radical can also be defined for non-unital rings.
To do this, we first define a binary operation $\circ$ on the ring $R$
by $x\circ y=x+y-xy$ for all $x,y\in R$.
Then $(R,\circ)$ is a monoid,
and the Jacobson radical is defined to be the largest ideal $I$ of $R$
such that $(I,\circ)$ is a group.
If $R$ is unital, this is equivalent to the definitions given earlier.
%%%%%
%%%%%
\end{document}
