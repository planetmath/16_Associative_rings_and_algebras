\documentclass[12pt]{article}
\usepackage{pmmeta}
\pmcanonicalname{OrdinaryQuiverOfAnAlgebra}
\pmcreated{2013-03-22 19:17:41}
\pmmodified{2013-03-22 19:17:41}
\pmowner{joking}{16130}
\pmmodifier{joking}{16130}
\pmtitle{ordinary quiver of an algebra}
\pmrecord{4}{42229}
\pmprivacy{1}
\pmauthor{joking}{16130}
\pmtype{Definition}
\pmcomment{trigger rebuild}
\pmclassification{msc}{16S99}
\pmclassification{msc}{20C99}
\pmclassification{msc}{13B99}

% this is the default PlanetMath preamble.  as your knowledge
% of TeX increases, you will probably want to edit this, but
% it should be fine as is for beginners.

% almost certainly you want these
\usepackage{amssymb}
\usepackage{amsmath}
\usepackage{amsfonts}

% used for TeXing text within eps files
%\usepackage{psfrag}
% need this for including graphics (\includegraphics)
%\usepackage{graphicx}
% for neatly defining theorems and propositions
%\usepackage{amsthm}
% making logically defined graphics
%%%\usepackage{xypic}

% there are many more packages, add them here as you need them

% define commands here

\begin{document}
Let $k$ be a field and $A$ an algebra over $k$.

Denote by $\mathrm{rad}A$ the (Jacobson) radical of $A$ and $\mathrm{rad}^2A=(\mathrm{rad}A)^2$ a square of radical.

Since $A$ is finite-dimensional, then we have a \PMlinkname{complete set of primitive orthogonal idempotents}{CompleteSetOfPrimitiveOrthogonalIdempotents} $E=\{e_1,\ldots,e_n\}$.

\textbf{Definition.} The \textbf{ordinary quiver} of a finite-dimensional algebra $A$ is defined as follows:
\begin{enumerate}
\item The set of vertices is equal to $Q_0=\{1,\ldots,n\}$ which is in bijective correspondence with $E$.
\item If $a,b\in Q_0$, then the number of arrows from $a$ to $b$ is equal to the dimension of the $k$-vector space
$$e_a\big(\mathrm{rad}A/\mathrm{rad}^2A\big)e_b.$$
\end{enumerate}

It can be shown that the ordinary quiver is well-defined, i.e. it is independent on the choice of a complete set of primitve orthogonal idempotents. Also finite dimension of $A$ implies, then the ordinary quiver is finite.
%%%%%
%%%%%
\end{document}
