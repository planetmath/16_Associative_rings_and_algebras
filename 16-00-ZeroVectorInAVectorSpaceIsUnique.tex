\documentclass[12pt]{article}
\usepackage{pmmeta}
\pmcanonicalname{ZeroVectorInAVectorSpaceIsUnique}
\pmcreated{2013-03-22 13:37:16}
\pmmodified{2013-03-22 13:37:16}
\pmowner{matte}{1858}
\pmmodifier{matte}{1858}
\pmtitle{zero vector in a vector space is unique}
\pmrecord{7}{34256}
\pmprivacy{1}
\pmauthor{matte}{1858}
\pmtype{Theorem}
\pmcomment{trigger rebuild}
\pmclassification{msc}{16-00}
\pmclassification{msc}{13-00}
\pmclassification{msc}{20-00}
\pmclassification{msc}{15-00}
\pmrelated{IdentityElementIsUnique}

% this is the default PlanetMath preamble.  as your knowledge
% of TeX increases, you will probably want to edit this, but
% it should be fine as is for beginners.

% almost certainly you want these
\usepackage{amssymb}
\usepackage{amsmath}
\usepackage{amsfonts}

% used for TeXing text within eps files
%\usepackage{psfrag}
% need this for including graphics (\includegraphics)
%\usepackage{graphicx}
% for neatly defining theorems and propositions
%\usepackage{amsthm}
% making logically defined graphics
%%%\usepackage{xypic}

% there are many more packages, add them here as you need them

% define commands here
\begin{document}
\PMlinkescapeword{satisfy}

{\bf Theorem} The zero vector in a vector space is unique.

\emph{Proof.} Suppose $0$ and $\tilde{0}$ are zero vectors
in a vector space $V$. Then both
$0$ and $\tilde{0}$ must satisfy
\PMlinkname{axiom 3}{VectorSpace},
i.e., for all $v\in V$,
\begin{eqnarray*}
v + 0 &=& v,\\
v + \tilde{0} &=& v.
\end{eqnarray*}
Setting $v=\tilde{0}$ in the first equation, and $v=0$
in the second yields
$\tilde{0} + 0 = \tilde{0}$ and
$0 + \tilde{0} = 0$. Thus, using
\PMlinkname{axiom 2}{VectorSpace},
\begin{eqnarray*}
{\displaystyle0} &= \tilde{0} + 0 \\
  &= 0 + \tilde{0} \\
  &= \tilde{0},
\end{eqnarray*}
and $0=\tilde{0}$. $\Box$
%%%%%
%%%%%
\end{document}
