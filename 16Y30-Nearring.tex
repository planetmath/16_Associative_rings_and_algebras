\documentclass[12pt]{article}
\usepackage{pmmeta}
\pmcanonicalname{Nearring}
\pmcreated{2013-03-22 13:25:12}
\pmmodified{2013-03-22 13:25:12}
\pmowner{yark}{2760}
\pmmodifier{yark}{2760}
\pmtitle{near-ring}
\pmrecord{22}{33968}
\pmprivacy{1}
\pmauthor{yark}{2760}
\pmtype{Definition}
\pmcomment{trigger rebuild}
\pmclassification{msc}{16Y30}
\pmsynonym{near ring}{Nearring}
\pmsynonym{nearring}{Nearring}
\pmrelated{Ring}
\pmdefines{commutative near-ring}
\pmdefines{commutative near ring}
\pmdefines{commutative nearring}
\pmdefines{distributative near-ring}
\pmdefines{distributative near ring}
\pmdefines{distributative nearring}
\pmdefines{near field}
\pmdefines{nearfield}

\endmetadata

\usepackage{amssymb}
\usepackage{amsmath}
\usepackage{amsfonts}

\begin{document}
\PMlinkescapeword{addition}
\PMlinkescapeword{additive}
\PMlinkescapeword{commutative}
\PMlinkescapeword{identity}
\PMlinkescapeword{inverse}
\PMlinkescapeword{natural}
\PMlinkescapeword{side}

\section*{Definitions}

A \emph{near-ring} is a \PMlinkname{set}{Set} $N$ together with two binary operations, denoted $+\colon N \times N \to N$ and $\cdot\colon N \times N \to N$, such that
\begin{enumerate}
\item $(a+b)+c = a+(b+c)$ and $(a \cdot b) \cdot c = a \cdot (b \cdot c)$ for all $a,b,c \in N$ (associativity of both operations)
\item There exists an element $0 \in N$ such that $a+0 = 0+a = a$ for all $a \in N$ (additive identity)
\item For all $a \in N$, there exists $b \in N$ such that $a+b = b+a = 0$ (additive inverse)
\item $(a+b) \cdot c = (a \cdot c) + (b \cdot c)$ for all $a,b,c \in N$ (right distributive law)
\end{enumerate}

Note that the axioms of a near-ring differ from those of a ring in that they do not require addition to be \PMlinkname{commutative}{Commutative}, and only require distributivity on one side.

A \emph{near-field} is a near-ring $N$
such that $(N\setminus\{0\},\cdot)$ is a group.

\section*{Notes}

Every element $a$ in a near-ring has a unique additive inverse, denoted $-a$.

We say $N$ has an {\em identity element} if there exists an element $1 \in N$ such that $a \cdot 1 = 1 \cdot a = a$ for all $a \in N$.
We say $N$ is {\em distributive} if $a\cdot(b+c) = (a \cdot b) + (a \cdot c)$ holds for all $a,b,c \in N$.
We say $N$ is {\em commutative} if $a \cdot b = b \cdot a$ for all $a,b \in N$.

Every commutative near-ring is distributive.
Every distributive near-ring with an identity element is a unital ring
(see the \PMlinkname{attached proof}{ConditionOnANearRingToBeARing}).

\section*{Example}

A natural example of a near-ring is the following. Let $(G,+)$ be a group (not necessarily \PMlinkname{abelian}{AbelianGroup2}), and let $M$ be the set of all functions from $G$ to $G$. For two functions $f$ and $g$ in $M$ define $f+g\in M$ by $(f+g)(x)=f(x)+g(x)$ for all $x\in G$. Then $(M,+,\circ)$ is a near-ring with identity, where $\circ$ denotes composition of functions.

\begin{thebibliography}{9}
\bibitem{pilz}
 G\"unter Pilz,
 {\it Near-Rings},
 North-Holland, 1983.
\end{thebibliography}
%%%%%
%%%%%
\end{document}
