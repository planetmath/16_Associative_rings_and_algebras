\documentclass[12pt]{article}
\usepackage{pmmeta}
\pmcanonicalname{QuotientModule}
\pmcreated{2013-03-22 14:01:18}
\pmmodified{2013-03-22 14:01:18}
\pmowner{rspuzio}{6075}
\pmmodifier{rspuzio}{6075}
\pmtitle{quotient module}
\pmrecord{9}{34966}
\pmprivacy{1}
\pmauthor{rspuzio}{6075}
\pmtype{Definition}
\pmcomment{trigger rebuild}
\pmclassification{msc}{16D10}
\pmdefines{quotient vector space}

\usepackage{amssymb}
\usepackage{amsmath}
\usepackage{amsfonts}
\begin{document}
Let $M$ be a module over a ring $R$, and let $S$ be a submodule of $M$.
The \emph{quotient module} $M/S$ is the quotient group $M/S$ with
scalar multiplication defined by $\lambda(x+S)=\lambda x+S$ for all
$\lambda\in R$ and all $x\in M$.  

This is a well defined operation. Indeed, if $x+S = x'+S$ then for
some $s\in S$ we have $x'=x+s$ and therefore
\begin{align*}
  \lambda x' &= \lambda(x+s)\\
                &= \lambda x+\lambda s
\end{align*}
so that $\lambda x' + S = \lambda x + \lambda s + S = \lambda x + S$,
since $\lambda s \in S$. 

In the special case that $R$ is a field this construction defines
the \emph{quotient vector space} of a vector space by a vector subspace.
%%%%%
%%%%%
\end{document}
