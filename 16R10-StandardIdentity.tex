\documentclass[12pt]{article}
\usepackage{pmmeta}
\pmcanonicalname{StandardIdentity}
\pmcreated{2013-03-22 14:21:10}
\pmmodified{2013-03-22 14:21:10}
\pmowner{CWoo}{3771}
\pmmodifier{CWoo}{3771}
\pmtitle{standard identity}
\pmrecord{7}{35830}
\pmprivacy{1}
\pmauthor{CWoo}{3771}
\pmtype{Definition}
\pmcomment{trigger rebuild}
\pmclassification{msc}{16R10}

\endmetadata

% this is the default PlanetMath preamble.  as your knowledge
% of TeX increases, you will probably want to edit this, but
% it should be fine as is for beginners.

% almost certainly you want these
\usepackage{amssymb,amscd}
\usepackage{amsmath}
\usepackage{amsfonts}

% used for TeXing text within eps files
%\usepackage{psfrag}
% need this for including graphics (\includegraphics)
%\usepackage{graphicx}
% for neatly defining theorems and propositions
%\usepackage{amsthm}
% making logically defined graphics
%%%\usepackage{xypic}

% there are many more packages, add them here as you need them

% define commands here
\begin{document}
Let $R$ be a commutative ring and $X$ be a set of non-commuting variables over $R$.  The \emph{standard identity of degree} $n$ in $R\langle X \rangle$, denoted by $[x_1,\ldots\,x_n]$, is the polynomial $$\sum_{\pi} \operatorname{sign}(\pi)x_{\pi(1)}\cdots x_{\pi(n)},\mbox{ where }\pi \in S_n.$$

\textbf{Remarks:}
\begin{itemize}
\item
A ring $R$ satisfying the standard identity of degree 2 (i.e., $[R,R]=0$) is commutative.  In this sense, algebras satisfying a standard identity is a generalization of the class of commutative algebras.
\item
Two immediate properties of $[x_1,\ldots\,x_n]$ are that it is \emph{multilinear} over $R$, and it is \emph{alternating}, in the sense that $[r_1,\ldots\,r_n]=0$ whenever two of the $r_i's$ are equal.  Because of these two properties, one can show that an n-dimensional algebra $R$ over a field $k$ is a PI-algebra, satisfying the standard identity of degree $n+1$.  As a corollary, $\mathbb{M}_n(k)$, the $n\times n$ matrix ring over a field $k$, is a PI-algebra satisfying the standard identity of degree $n^2+1$.  In fact, Amitsur and Levitski have shown that $\mathbb{M}_n(k)$ actually satisfies the standard identity of degree $2n$.
\end{itemize}

\begin{thebibliography}{7}
\bibitem{amitsur} S. A. Amitsur and J. Levitski, {\em Minimal identities for algebras}, Proc. Amer. Math. Soc., 1 (1950) 449-463.
\end{thebibliography}
%%%%%
%%%%%
\end{document}
