\documentclass[12pt]{article}
\usepackage{pmmeta}
\pmcanonicalname{partialTiltingModule}
\pmcreated{2013-12-11 16:01:05}
\pmmodified{2013-12-11 16:01:05}
\pmowner{joking}{16130}
\pmmodifier{joking}{16130}
\pmtitle{(partial) tilting module}
\pmrecord{8}{42245}
\pmprivacy{1}
\pmauthor{joking}{16130}
\pmtype{Definition}
\pmcomment{trigger rebuild}
\pmclassification{msc}{16S99}
\pmclassification{msc}{20C99}
\pmclassification{msc}{13B99}

\endmetadata

% this is the default PlanetMath preamble.  as your knowledge
% of TeX increases, you will probably want to edit this, but
% it should be fine as is for beginners.

% almost certainly you want these
\usepackage{amssymb}
\usepackage{amsmath}
\usepackage{amsfonts}

% used for TeXing text within eps files
%\usepackage{psfrag}
% need this for including graphics (\includegraphics)
%\usepackage{graphicx}
% for neatly defining theorems and propositions
%\usepackage{amsthm}
% making logically defined graphics
%%\usepackage{xypic}

% there are many more packages, add them here as you need them

% define commands here

\begin{document}
Let $A$ be an associative, finite-dimensional algebra over a field $k$. Throughout all modules are finite-dimensional.

A right $A$-module $T$ is called a \textbf{partial tilting module} if the projective dimension of $T$ is at most $1$ ($\mathrm{pd}T\leqslant 1$) and $\mathrm{Ext}^1_A(T,T)=0$.

Recall that if $M$ is an $A$-module, then by $\mathrm{add}M$ we denote the class of all $A$-modules which are direct sums of direct summands of $M$. Since Krull-Schmidt Theorem holds in the category of finite-dimensional $A$-modules, then this means, that if
$$M=E_1\oplus\cdots\oplus E_n$$
for some indecomposable modules $E_i$, then $\mathrm{add}M$ consists of all modules which are isomorphic to
$$E_1^{a_1}\oplus\cdots\oplus E_n^{a_n}$$
for some nonnegative integers $a_1,\ldots, a_n$.

A partial tiliting module $T$ is called a \textbf{tilting module} if there exists a short exact sequence
$$
0 \rightarrow A \rightarrow T' \rightarrow T'' \rightarrow 0
$$
such that both $T',T''\in\mathrm{add}T$. Here we treat the algebra $A$ is a right module via multiplication.

Note that every projective module is partial tilting. Also a projective module $P$ is tilting if and only if every indecomposable direct summand of $A$ is a direct summand of $P$.
%%%%%
%%%%%
\end{document}
