\documentclass[12pt]{article}
\usepackage{pmmeta}
\pmcanonicalname{ExampleOfAnArtinianModuleWhichIsNotNoetherian}
\pmcreated{2013-03-22 19:04:18}
\pmmodified{2013-03-22 19:04:18}
\pmowner{joking}{16130}
\pmmodifier{joking}{16130}
\pmtitle{example of an Artinian module which is not Noetherian}
\pmrecord{5}{41956}
\pmprivacy{1}
\pmauthor{joking}{16130}
\pmtype{Example}
\pmcomment{trigger rebuild}
\pmclassification{msc}{16D10}

% this is the default PlanetMath preamble.  as your knowledge
% of TeX increases, you will probably want to edit this, but
% it should be fine as is for beginners.

% almost certainly you want these
\usepackage{amssymb}
\usepackage{amsmath}
\usepackage{amsfonts}

% used for TeXing text within eps files
%\usepackage{psfrag}
% need this for including graphics (\includegraphics)
%\usepackage{graphicx}
% for neatly defining theorems and propositions
%\usepackage{amsthm}
% making logically defined graphics
%%%\usepackage{xypic}

% there are many more packages, add them here as you need them

% define commands here

\begin{document}
It is well known, that left (right) Artinian ring is left (right) Noetherian (Akizuki-Hopkins-Levitzki theorem). We will show that this no longer holds for modules.

Let $\mathbb{Z}$ be the ring of integers and $\mathbb{Q}$ the field of rationals. Let $p\in\mathbb{Z}$ be a prime number and consider 
$$G=\{\frac{a}{p^n}\in\mathbb{Q}\ |\ a\in\mathbb{Z};\ n\geq 0\}.$$
Of course $G$ is a $\mathbb{Z}$-module via standard multiplication and addition. For $n\geq 0$ consider
$$G_n=\{\frac{a}{p^n}\in\mathbb{Q}\ |\ a\in\mathbb{Z}\}.$$
Of course each $G_n\subseteq G$ is a submodule and it is easy to see, that
$$\mathbb{Z}=G_0\subset G_1\subset G_2\subset G_3\subset\cdots,$$
where each inclusion is proper. We will show that $G/\mathbb{Z}$ is Artinian, but it is not Noetherian.

Let $\pi:G\to G/\mathbb{Z}$ be the canonical projection. Then $G'_n=\pi(G_n)$ is a submodule of $G/\mathbb{Z}$ and
$$0=G'_0\subset G'_1\subset G'_2\subset G'_3\subset G'_4\subset\cdots.$$
The inclusions are proper, because for any $n>0$ we have 
$$G'_{n+1}/G'_{n}\simeq \big(G_{n+1}/\mathbb{Z}\big)/\big(G_{n}/\mathbb{Z}\big)\simeq G_{n+1}/G_n\neq 0,$$
due to Third Isomorphism Theorem for modules. This shows, that $G/\mathbb{Z}$ is not Noetherian.

In order to show that $G/\mathbb{Z}$ is Artinian, we will show, that each proper submodule of $G/\mathbb{Z}$ is of the form $G'_n$. Let $N\subseteq G/\mathbb{Z}$ be a proper submodule. Assume that for some $a\in\mathbb{Z}$ and $n\geq 0$ we have 
$$\frac{a}{p^n}+\mathbb{Z}\in N.$$
We may assume that $\mathrm{gcd}(a,p^n)=1$. Therefore there are $\alpha,\beta\in\mathbb{Z}$ such that
$$1=\alpha a+\beta p^n.$$
Now, since $N$ is a $\mathbb{Z}$-module we have
$$\frac{\alpha a}{p^n}+\mathbb{Z}\in N$$
and since $0+\mathbb{Z}=\beta+\mathbb{Z}=\frac{\beta p^n}{p^n}+\mathbb{Z}\in N$ we have that
$$\frac{1}{p^n}+\mathbb{Z}=\frac{\alpha a +\beta p^n}{p^n}+\mathbb{Z}\in N.$$
Now, let $m> 0$ be the smallest number, such that $\frac{1}{p^m}+\mathbb{Z}\not\in N$. What we showed is that
$$N=G'_{m-1}=\pi(G_{m-1}),$$
because for every $0\leq n\leq m-1$ (and only for such $n$) we have $\frac{1}{p^n}+\mathbb{Z}\in N$ and thus $N$ is a image of a submodule of $G$, which is generated by $\frac{1}{p^n}$ and this is precisely $G_{m-1}$. Now let
$$N_1\supseteq N_2\supseteq N_3\supseteq\cdots$$
be a chain of submodules in $G/\mathbb{Z}$. Then there are natural numbers $n_1,n_2,\ldots$ such that $N_i=G'_{n_i}$. Note that $G'_{k}\supseteq G'_{s}$ if and only if $k\geq s$. In particular we obtain a sequence of natural numbers
$$n_1\geq n_2\geq n_3\geq\cdots$$
This chain has to stabilize, which completes the proof. $\square$
%%%%%
%%%%%
\end{document}
