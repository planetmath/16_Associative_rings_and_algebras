\documentclass[12pt]{article}
\usepackage{pmmeta}
\pmcanonicalname{ProofOfProductOfLeftAndRightIdeal}
\pmcreated{2013-03-22 17:41:25}
\pmmodified{2013-03-22 17:41:25}
\pmowner{rm50}{10146}
\pmmodifier{rm50}{10146}
\pmtitle{proof of product of left and right ideal}
\pmrecord{6}{40131}
\pmprivacy{1}
\pmauthor{rm50}{10146}
\pmtype{Proof}
\pmcomment{trigger rebuild}
\pmclassification{msc}{16D25}

\endmetadata

% this is the default PlanetMath preamble.  as your knowledge
% of TeX increases, you will probably want to edit this, but
% it should be fine as is for beginners.

% almost certainly you want these
\usepackage{amssymb}
\usepackage{amsmath}
\usepackage{amsfonts}

% used for TeXing text within eps files
%\usepackage{psfrag}
% need this for including graphics (\includegraphics)
%\usepackage{graphicx}
% for neatly defining theorems and propositions
%\usepackage{amsthm}
% making logically defined graphics
%%%\usepackage{xypic}

% there are many more packages, add them here as you need them

% define commands here
\newtheorem{thm}{Theorem}

\begin{document}
\begin{thm} Let $ \mathfrak{a}$ and $ \mathfrak{b}$ be ideals of a ring $ R$. Denote by $ \mathfrak{ab}$ the subset of $ R$ formed by all finite sums of products $ ab$ with $ a \in \mathfrak{a}$ and $ b \in \mathfrak{b}$. Then if $ \mathfrak{a}$ is a left and $ \mathfrak{b}$ a right ideal, $ \mathfrak{ab}$ is a two-sided ideal of $ R$. If in addition both $ \mathfrak{a}$ and $ \mathfrak{b}$ are two-sided ideals, then $ \mathfrak{ab} \subseteq \mathfrak{a}\cap\mathfrak{b}$.
\end{thm}

\textbf{Proof.} 
We must show that the difference of any two elements of $\mathfrak{ab}$ is in $\mathfrak{ab}$, and that $\mathfrak{ab}$ is closed under multiplication by $R$. But both of these operations are linear in $\mathfrak{ab}$; that is, if they hold for elements of the form $ab, a\in\mathfrak{a}, b\in\mathfrak{b}$, then they hold for the general element of $\mathfrak{ab}$. So we restrict our analysis to elements $ab$.

Clearly if $a_1,a_2\in \mathfrak{a}, b_1,b_2\in\mathfrak{b}$, then $a_1b_1-a_2b_2\in\mathfrak{ab}$ by definition. 

If $a\in\mathfrak{a}, b\in\mathfrak{b}, r\in R$, then
\begin{gather*}
r \cdot ab = (r\cdot a)b\in\mathfrak{ab} \text{ since } \mathfrak{a} \text{ is a left ideal} \\
ab \cdot r = a(b\cdot r)\in\mathfrak{ab} \text{ since } \mathfrak{b} \text{ is a right ideal}
\end{gather*}
and thus $\mathfrak{ab}$ is a two-sided ideal. This proves the first statement.

If $\mathfrak{a},\mathfrak{b}$ are two-sided ideals, then $ab\in\mathfrak{a}$ since $b\in R$; similarly, $ab\in\mathfrak{b}$. This proves the second statement.
%%%%%
%%%%%
\end{document}
