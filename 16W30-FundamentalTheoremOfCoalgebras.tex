\documentclass[12pt]{article}
\usepackage{pmmeta}
\pmcanonicalname{FundamentalTheoremOfCoalgebras}
\pmcreated{2013-03-22 18:49:22}
\pmmodified{2013-03-22 18:49:22}
\pmowner{joking}{16130}
\pmmodifier{joking}{16130}
\pmtitle{fundamental theorem of coalgebras}
\pmrecord{6}{41625}
\pmprivacy{1}
\pmauthor{joking}{16130}
\pmtype{Theorem}
\pmcomment{trigger rebuild}
\pmclassification{msc}{16W30}

\endmetadata

% this is the default PlanetMath preamble.  as your knowledge
% of TeX increases, you will probably want to edit this, but
% it should be fine as is for beginners.

% almost certainly you want these
\usepackage{amssymb}
\usepackage{amsmath}
\usepackage{amsfonts}

% used for TeXing text within eps files
%\usepackage{psfrag}
% need this for including graphics (\includegraphics)
%\usepackage{graphicx}
% for neatly defining theorems and propositions
%\usepackage{amsthm}
% making logically defined graphics
%%%\usepackage{xypic}

% there are many more packages, add them here as you need them

% define commands here

\begin{document}
\textbf{Fundamental Theorem of Coalgebras.} Let $(C,\Delta,\varepsilon)$ be a coalgebra over a field $k$ and $x\in C$. Then there exists subcoalgebra $D\subseteq C$ such that $x\in D$ and $\mathrm{dim}_{k}\,D<\infty$.

\textit{Proof.} Let $$\Delta(x)=\sum_{i} b_i\otimes c_i.$$ Consider the element $$\Delta_2(x)=\sum_{i} \Delta(b_i)\otimes c_i = \sum_{i,j} a_j\otimes b_{ij}\otimes c_i.$$ Note that we may assume that $(a_j)$ are linearly independent and so are $(c_i)$. Let $D$ be a subspace spanned by $(b_{ij})$. Of course $\mathrm{dim}_{k}\,D<\infty$. Furthermore $x\in D$, because $$x=\sum_{i,j} \varepsilon(a_j)\varepsilon(c_i)b_{ij}.$$ We will show that $D$ is a subcoalgebra, i.e. $\Delta(D)\subseteq D\otimes D$. Indeed, note that $$\sum_{i,j} \Delta(a_j)\otimes b_{ij}\otimes c_i=\sum_{i,j} a_j\otimes \Delta(b_{ij})\otimes c_i$$ and since $c_i$ are linearly independent we obtain that $$\sum_{j} \Delta(a_j)\otimes b_{ij}=\sum_{j} a_j\otimes \Delta(b_{ij})$$ for all $i$. Thus $$\sum_{j} a_j\otimes \Delta(b_{ij})\in C\otimes C\otimes D$$ and since $a_j$ are linearly independent, we obtain that $\Delta(b_{ij})\in C\otimes D$ for all $i,j$. Analogously we show that $\Delta(b_{ij})\in D\otimes C$, thus $$\Delta(b_{ij})\in C\otimes D\cap D\otimes C= D\otimes D,$$ (please, see \PMlinkname{this entry}{TensorProductOfSubspacesOfVectorSpaces} for last equality) which completes the proof. $\square$

\textbf{Remark.} The category of finite dimensional coalgebras is dual to the category of finite dimensional algebras (via dual space functor), so one could think that generally they are similar. Unfortunetly Fundamental Theorem of Coalgebras is major diffrence between algebras and coalgebras. For example consider a field $k$ and its polynomial algebra $k[X]$. Then whenever $f\in k[X]$ is such that $\mathrm{deg}\,(f)>0$, then a subalgebra generated by $f$ is always infinite dimensional (if $\mathrm{deg}\,(f)=0$ then subalgebra generated by $f$ is $k$). This can never occur in coalgebras.
%%%%%
%%%%%
\end{document}
